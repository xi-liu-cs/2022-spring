\documentclass[12pt,border=4pt,multi]{article} % \documentclass[tikz,border=4pt,multi]{article}
\usepackage{lingmacros}
\usepackage{tree-dvips}
\usepackage{amssymb} % for mathbb{}
\usepackage[dvipsnames]{xcolor}
\usepackage{forest}
\usepackage{amsmath} % for matrices
\usepackage{xeCJK}
\usepackage{tikz}
\usepackage[arrowdel]{physics}
\usepackage{graphicx}
\usepackage{wrapfig}
\usepackage{listings}
\usepackage{pgfplots, pgfplotstable}
\usepackage{diagbox} % diagonal line in cell
\usepackage[usestackEOL]{stackengine}
\usepackage{multirow}
\graphicspath{{./img}} % specify the graphics path to be relative to the main .tex file, denoting the main .tex file directory as ./
\newcommand\Myperm[2][^n]{\prescript{#1\mkern-2.5mu}{}P_{#2}}
\newcommand\Mycomb[2][^n]{\prescript{#1\mkern-0.5mu}{}C_{#2}}
\definecolor{orchid}{rgb}{0.7, 0.4, 1.1}

\begin{document}

\section*{Xi Liu, xl3504, Problem Set 7}
Problem 1\\
1.
\begin{align*}
P(X = 1, Y = 1, Z = 1 - 2(1) = -1) &= \frac{1}{3}\\
P(X = 1, Y = 2, Z = 1 - 2(2) = -3) &= \frac{1}{12}\\
P(X = 2, Y = 1, Z = 2 - 2(1) = 0) &= \frac{1}{6}\\
P(X = 2, Y = 2, Z = 2 - 2(2) = -2) &= 0\\
P(X = 4, Y = 1, Z = 4 - 2(1) = 2) &= \frac{1}{12}\\
P(X = 4, Y = 2, Z = 4 - 2(2) = 0) &= \frac{1}{3}\\
\end{align*}
\[P(Z = 0) = \frac{1}{6} + \frac{1}{3} = \frac{1 + 2}{6} = \frac{1}{2}\]
{\Large
\begin{center}
\begin{tabular}{|c|c|c|c|c|c|}\hline
$a$ & -3 & -2 & -1 & 0 & 2\\ \hline
$p(Z = a)$ & $\frac{1}{12}$ & 0 & $\frac{1}{3}$ & $\frac{1}{2}$ & $\frac{1}{12}$\\ \hline
\end{tabular}
\end{center}
}
\newpage
\noindent
2.
\[g(x, y) = x - 2y\]
\begin{align*}
E[Z] &= E[g(X, Y)] = \sum_j \sum_i g(a_i, b_j) p_{X, Y}(a_i, b_j)\\
&= \sum_j \sum_i g(a_i, b_j) p(X = a_i, Y = b_j)\\
&= g(1, 1)p(1, 1) + g(1, 2)p(1, 2) + g(2, 1)p(2, 1)\\
&\quad + g(2, 2)p(2, 2) + g(4, 1)p(4, 1) + g(4, 2)p(4, 2)\\
&= -1 \cdot \frac{1}{3} + -3 \cdot \frac{1}{12} + 0 \cdot \frac{1}{6}
+ -2 \cdot 0 + 2 \cdot \frac{1}{12} + 0 \cdot \frac{1}{3}\\
&= \boxed{-\frac{5}{12}}\\
\end{align*}
\\
\\
\\
\\
3.
\begin{align*}
P(X = 2 | Z = 0) &= \frac{P(X = 2, Z = 0)}{P(Z = 0)}\\
&= \frac{P(X = 2, Y = 1, Z = 2 - 2(1) = 0)}{P(Z = 0)}\\
&= \frac{1/6}{1/2}\\
&= \frac{2}{6}\\
&= \boxed{\frac{1}{3}}\\
\end{align*}
\newpage
\noindent
Problem 2
\begin{align*}
p_N(n) &= P(N = n) = \frac{\lambda^n}{n!}e^{-\lambda}\\
\end{align*}
$P(X = x|N = n) = binomial(n, p) =\\
{\large
\begin{cases}
\binom{n}{x} p^x (1 - p)^{n - x} & \text{if } x \in [0, n] \cap \mathbb{N}\\
0 & \text{otherwise}\\
\end{cases}}$
\\
\begin{align*}
P(X = x|N = n) &= \frac{P(X = x, N = n)}{P(N = n)}\\
P(X = x, N = n) &= P(X = x|N = n)P(N = n)\\
&= \binom{n}{x} p^x (1 - p)^{n - x} P(N = n)\\
&= \boxed{\binom{n}{x} p^x (1 - p)^{n - x} \frac{\lambda^n}{n!}e^{-\lambda}}\\
\end{align*}
\newpage
\noindent
Problem 3
\begin{center}
\begin{tabular}{|c|c|c|c|}\hline
\diagbox[width = 2cm, height = 1cm]{$X$}{$Y$} & 1 & 2 & 3\\ \hline
1 & $\Centerstack{p(1,1) = 1/9\\ W = 2\\ Z = 0}$ & $\Centerstack{p(1,2) = 1/9\\ W = 3\\ Z = -1}$ & $\Centerstack{p(1,3) = 1/9\\ W = 4\\ Z = -2}$\\ \hline
2 & $\Centerstack{p(2,1) = 1/9\\ W = 3\\ Z = 1}$ & $\Centerstack{p(2,2) = 1/9\\ W = 4\\ Z = 0}$ & $\Centerstack{p(2,3) = 1/9\\ W = 5\\ Z = -1}$\\ \hline
3 & $\Centerstack{p(3,1) = 1/9\\ W = 4\\ Z = 2}$ & $\Centerstack{p(3,2) = 1/9\\ W = 5\\ Z = 1}$ & $\Centerstack{p(3,3) = 1/9\\ W = 6\\ Z = 0}$\\ \hline
\end{tabular}
\end{center}
\leavevmode
\\
\\
\\
1.\\
joint probability mass function of $W$ and $Z$:
\begin{center}
\begin{tabular}{|c|c|c|c|c|c|}\hline
\diagbox[width = 2cm, height = 1cm]{$W$}{$Z$} & -2 & -1 & 0 & 1 & 2\\ \hline
2 & 0 & 0 & 1/9 & 0 & 0\\ \hline
3 & 0 & 1/9 & 0 & 1/9 & 0\\ \hline
4 & 1/9 & 0 & 1/9 & 0 & 1/9\\ \hline
5 & 0 & 1/9 & 0 & 1/9 & 0\\ \hline
6 & 0 & 0 & 1/9 & 0 & 0\\ \hline
\end{tabular}
\end{center}
\leavevmode
\\
\\
\\
\\
2.\\
\\
2 discrete random variables $X$ and $Y$ are independent if 
\[P(X = a, Y = b) = P(X = a)P(Y = b)\]
\\
$W$ and $Z$ are not independent, since for example, $P(W = 2) = 1/9$,\\
$P(Z = 0) = 3/9$,\\
$P(W = 2, Z = 0) = 1/9 \not= P(W = 2)P(Z = 0) = (1/9)(3/9) = 1/27$
\\
\\
\\
\\
3.
\[g(x, y) := x + y\]
\begin{align*}
E[W] &= E[g(X, Y)]\\
&= \sum_j \sum_i g(a_i, b_j) p_{X, Y}(a_i, b_j)\\
&= \sum_j \sum_i g(a_i, b_j) p(X = a_i, Y = b_j)\\
&= g(1,1)p(1,1) + g(1,2)p(1,2) + g(1,3)p(1,3)\\
&\quad+ g(2,1)p(2,1) + g(2,2)p(2,2) + g(2,3)p(2,3)\\
&\quad+ g(3,1)p(3,1) + g(3,2)p(3,2) + g(3,3)p(3,3)\\
&= 2(1/9) + 3(1/9) + 4(1/9)\\
&\quad+ 3(1/9) + 4(1/9) + 5(1/9)\\
&\quad+ 4(1/9) + 5(1/9) + 6(1/9)\\
&= (2 + 3 + 4 + 3 + 4 + 5 + 4 + 5 + 6)(1/9)\\
&= \boxed{4}\\
\end{align*}
\begin{align*}
&\quad\quad h(x, y) := x - y\\
\\
E[Z] &= E[h(X, Y)]\\
&= \sum_j \sum_i h(a_i, b_j) p_{X, Y}(a_i, b_j)\\
&= \sum_j \sum_i h(a_i, b_j) p(X = a_i, Y = b_j)\\
&= h(1,1)p(1,1) + h(1,2)p(1,2) + h(1,3)p(1,3)\\
&\quad+ h(2,1)p(2,1) + h(2,2)p(2,2) + h(2,3)p(2,3)\\
&\quad+ h(3,1)p(3,1) + h(3,2)p(3,2) + h(3,3)p(3,3)\\
&= 0(1/9) + -1(1/9) + -2(1/9)\\
&\quad+ 1(1/9) + 0(1/9) + -1(1/9)\\
&\quad+ 2(1/9) + 1(1/9) + 0(1/9)\\
&= (- 1 - 2 + 1 - 1 + 2 + 1)(1/9)\\
&= \boxed{0}\\
\end{align*}
\newpage
\noindent
Problem 4\\
1.
\begin{align*}
f_X(i) &= \int_{-\infty}^{\infty} f_{X, Y}(i, y) dy\\
&= \int_{0}^{\infty} abe^{-ai - by} dy\\
&= ab\int_{0}^{\infty} e^{-ai - by} dy\\
&\text{/* } u:= -ai - by;\quad du = -bdy;\quad dy = -\frac{du}{b}\text{ */}\\
&= -a\left[e^{-ai - by}\right]_0^{y = \infty}\\
&= -a(0 - e^{-ai - 0})\\
&= \boxed{ae^{-ai}}\\
\end{align*}
\begin{align*}
f_Y(j) &= \int_{-\infty}^{\infty} f_{X, Y}(x, j) dx\\
&= \int_{0}^{\infty} abe^{-ax - bj} dx\\
&= ab\int_{0}^{\infty} e^{-ax - bj} dx\\
&\text{/* } u:= -ax - bj;\quad du = -adx;\quad dx = -\frac{du}{a}\text{ */}\\
&= -b\left[e^{-ax - bj}\right]_0^{\infty}\\
&= -b(0 - (e^{0 - bj}))\\
&= \boxed{be^{-bj}}\\
\end{align*}
\newpage
\noindent
2.
\begin{align*}
E[X] &= \int_{-\infty}^{\infty} x f_X(x) dx\\
&= \int_{0}^{\infty} x (ae^{-ax}) dx\\
&= a\int_{0}^{\infty} xe^{-ax} dx\\
&\text{/* } u:= x;\quad dv := e^{-ax};\\
&du = dx;\quad v = -\frac{1}{a}e^{-ax}\text{ */}\\
&= a\left[-\frac{x}{a}e^{-ax} - \frac{1}{a^2}e^{-ax}\right]_0^{\infty}\\
&= a\left(-\frac{1}{a^2}\right)\\
&= \boxed{-\frac{1}{a}}\\
\end{align*}
\begin{align*}
E[Y] &= \int_{-\infty}^{\infty} y f_Y(y) dy\\
&= \int_{0}^{\infty} y (be^{-by}) dy\\
&= b\int_{0}^{\infty} y e^{-by} dy\\
&\text{/* } u:= y;\quad dv := e^{-by}dy;\\
&du = dy;\quad v = -\frac{1}{b}e^{-by}\text{ */}\\
&= b\left[-\frac{y}{b}e^{-by} - \frac{1}{b^2}e^{-by}\right]_0^{\infty}\\
&= b\left(-\frac{1}{b^2}\right)\\\
&= \boxed{-\frac{1}{b}}\\
\end{align*}
\newpage
\noindent
3.
\begin{align*}
P(X < Y) &= \int_{0}^{\infty} \int_{0}^{y} f_{X, Y}(x, y) dx dy\\
&= \int_{0}^{\infty} \int_{0}^{y} (abe^{-ax - by}) dx dy\\
&= ab\int_{0}^{\infty} \int_{0}^{y} e^{-ax - by} dx dy\\
&\text{/* } u := - ax - by;\quad du = -adx;\quad dx = -\frac{du}{a}\text{ */}\\
&= -b\int_{0}^{\infty} \left[e^{-ax - by}\right]_{x = 0}^{x = y} dy\\
&= -b\int_{0}^{\infty} \left(e^{-(a + b)y} - e^{-by}\right) dy\\
&= -b\left[-\frac{1}{a + b}e^{-(a + b)y} + \frac{1}{b}e^{-by}\right]_0^{\infty}\\
&= -b\left(0 - \left(-\frac{1}{a + b} + \frac{1}{b}\right)\right)\\
&= b\left(-\frac{1}{a + b} + \frac{1}{b}\right)\\
&= \boxed{-\frac{b}{a + b} + 1}\\
\end{align*}
\newpage
\noindent
Problem 5
\begin{align*}
f_X(a) &= \int_{-\infty}^{\infty} f_{X, Y}(a, y)dy\\
&= \int_{a}^{\infty} e^{-y} dy\\
&= -\left[e^{-y}\right]_a^{\infty}\\
&= -(0 - e^{-a})\\
&= e^{-a}\\
\end{align*}
\begin{align*}
F_X(a) &= \int_{-\infty}^{a} f_X(x) dx\\
&= \int_{0}^{a} e^{-x} dx\\
&= -[e^{-x}]_0^a\\
&= -(e^{-a} - e^0)\\
&= 1 - e^{-a}\\
\end{align*}
\begin{align*}
f_Y(b) &= \int_{-\infty}^{\infty} f_{X, Y}(x, b)dx\\ 
&= \int_{0}^{b} e^{-b} dx\\
&= [xe^{-b}]_0^{b}\\
&= be^{-b}\\
\end{align*}
\begin{align*}
F_Y(b) &= \int_{-\infty}^{b} f_Y(y) dy\\
&= \int_{0}^{b} ye^{-y} dy\\
&\text{/* } u:= y;\quad dv := e^{-y}dy;\\
&du = dy;\quad v = -e^{-y}\text{ */}\\
&= [-ye^{-y}]_0^b - \int_0^b (-e^{-y}) dy\\
&= -be^{-b} + \int_0^b e^{-y} dy\\
&= -be^{-b} - [e^{-y}]_0^b\\
&= -be^{-b} - (e^{-b} - e^{0})\\
&= -be^{-b} - e^{-b} + 1\\
\end{align*}
if $a < b$
\begin{align*}
F_{X, Y}(a, b) &= \int_{0}^{a} \int_{x}^b f_{X, Y}(x, y) dy dx\\
&= \int_{0}^{a} \int_{x}^b e^{-y} dy dx\\
&= -\int_{0}^{a} [e^{-y}]_{y = x}^{y = b} dx\\
&= -\int_{0}^{a} (e^{-b} - e^{-x}) dx\\
&= -[xe^{-b} + e^{-x}]_0^a\\
&= -(ae^{-b} + e^{-a} - 1)\\
&= 1 - ae^{-b} - e^{-a}\\
\end{align*}
if $a \geq b$
\begin{align*}
F_{X, Y}(a, b) &= \int_{0}^{b} \int_{x}^b f_{X, Y}(x, y) dy dx\\
&= \int_{0}^{b} \int_{x}^b e^{-y} dy dx\\
&= -\int_{0}^{b} [e^{-y}]_{y = x}^{y = b} dx\\
&= -\int_{0}^{b} (e^{-b} - e^{-x}) dx\\
&= -[xe^{-b} + e^{-x}]_0^b\\
&= -(be^{-b} + e^{-b} - 1)\\
&= 1 - be^{-b} - e^{-b}\\
\end{align*}
\begin{center}
$P(X \leq a, Y \leq b) =
F_{X, Y}(a, b) =
\begin{cases}
1 - ae^{-b} - e^{-a} & \text{if } a < b\\
1 - be^{-b} - e^{-b} & \text{if } a \geq b\\
\end{cases}$
\end{center}
\begin{align*}
P(X \leq a)P(Y \leq b) &= F_X(a)F_Y(b) = (1 - e^{-a})(-be^{-b} - e^{-b} + 1)\\
&= -be^{-b} - e^{-b} + 1 + be^{-a}e^{-b} + e^{-a}e^{-b} - e^{-a}\\
\\
P(X \leq a, Y \leq b) &\not= P(X \leq a)P(Y \leq b)\\
\end{align*}
so $X$ and $Y$ are not independent\\
\end{document}