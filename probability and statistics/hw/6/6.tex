\documentclass[12pt,border=4pt,multi]{article} % \documentclass[tikz,border=4pt,multi]{article}
\usepackage{lingmacros}
\usepackage{tree-dvips}
\usepackage{amssymb} % for mathbb{}
\usepackage[dvipsnames]{xcolor}
\usepackage{forest}
\usepackage{amsmath} % for matrices
\usepackage{xeCJK}
\usepackage{tikz}
\usepackage[arrowdel]{physics}
\usepackage{graphicx}
\usepackage{wrapfig}
\usepackage{listings}
\usepackage{pgfplots, pgfplotstable}
\usepackage{diagbox} % diagonal line in cell
\usepackage[usestackEOL]{stackengine}
\usepackage{multirow}
\graphicspath{{./img}} % specify the graphics path to be relative to the main .tex file, denoting the main .tex file directory as ./
\newcommand\Myperm[2][^n]{\prescript{#1\mkern-2.5mu}{}P_{#2}}
\newcommand\Mycomb[2][^n]{\prescript{#1\mkern-0.5mu}{}C_{#2}}
\definecolor{orchid}{rgb}{0.7, 0.4, 1.1}

\begin{document}

\section*{Xi Liu, xl3504, Problem Set 6}
Problem 1\\
1.
\begin{center}
\begin{tabular}{|c|c|c|c|c|c|c|}\hline
\diagbox[width = 2cm, height = 1cm]{$X$}{$Y$} & -2 & 5 & 8 & $\not\in \{-2,\;5,\;8\}$ & & $p_X(x)$\\ \hline
1 & $\Centerstack{(0.7)(0.3) \\ = 0.21}$ & $\Centerstack{(0.7)(0.5) \\ = 0.35}$ & $\Centerstack{(0.7)(0.2) \\ = 0.14}$ & $\Centerstack{(0.7)(0) \\ = 0}$ & & 0.7\\ \hline
2 & $\Centerstack{(0.3)(0.3) \\ = 0.09}$ & $\Centerstack{(0.3)(0.5) \\ = 0.15}$ & $\Centerstack{(0.3)(0.2) \\ = 0.06}$ & $\Centerstack{(0.3)(0) \\ = 0}$ & & 0.3\\ \hline
$\not\in \{1,\;2\}$ & $\Centerstack{(0)(0.3) \\ = 0}$ & $\Centerstack{(0)(0.5) \\ = 0}$ & $\Centerstack{(0)(0.2) \\ = 0}$ & $\Centerstack{(0.3)(0) \\ = 0}$ & & 0\\ \hline
& & & & & &\\ \hline
$p_Y(y)$ & 0.3 & 0.5 & 0.2 & 0 & & 1\\ \hline
\end{tabular}
\end{center}
\leavevmode
\\
\\
\\
\\
2.
\begin{align*}
    P(X \text{ is even},\; Y \text{ is even}) &= p_{X, Y}(2, -2) + p_{X, Y}(2, 8)\\
    &= 0.09 + 0.06\\
    &= \boxed{0.15}\\
\end{align*}
\\
\\
\\
3.
\begin{align*}
    P(X = 1 \;|\; Y > 0) &= \frac{P(X = 1,\; Y > 0)}{P(Y > 0)}\\
    &= \frac{p_{X, Y}(1, 5) + p_{X, Y}(1, 8)}{p_{Y}(5) + p_{Y}(8)}\\
    &= \frac{0.35 + 0.14}{0.5 + 0.2}\\
    &= \frac{0.49}{0.7}\\
    &= \boxed{0.7}\\
\end{align*}
Problem 2\\
1.\\
$X$ = number of diamonds picked\\
$Y$ = number of queens picked\\
\\
in a standard deck of 52 cards\\
number of diamonds = 13\\
number of queens = 4\\
{\Large
\begin{center}
\begin{tabular}{|c|c|c|c|c|c|} \hline
\diagbox[width = 2cm, height = 1cm]{$X$}{$Y$} & 0 & 1 & 2 & & $p_X(x)$\\ \hline
0 & $\frac{105}{221}$ & $\frac{18}{221}$ & $\frac{1}{442}$ & \;\; & $\frac{19}{34}$\\ \hline
1 & $\frac{72}{221}$ & $\frac{12}{221}$ & $\frac{1}{442}$ & & $\frac{13}{34}$\\ \hline
2 & $\frac{11}{221}$ & $\frac{2}{221}$ & 0 & & $\frac{1}{17}$\\ \hline
& & & & &\\ \hline
$p_Y(y)$ & $\frac{188}{221}$ & $\frac{32}{221}$ & $\frac{1}{221}$ & & 1\\ \hline
\end{tabular}
\end{center}
}
\leavevmode
let $1st$ denote first card picked, $2nd$ denote second card picked, $d$ denote diamond, $q$ denote queen
\begin{align*}
p_{X, Y}(0, 0) &= P(1st \not= d,\; 1st \not= q,\; 2nd \not= d,\; 2nd \not= q)\\
&= \frac{52 - 13 - 3}{52} \cdot \frac{51 - 13 - 3}{51} = \boxed{\frac{105}{221}}\\
p_{X, Y}(0, 1) &= P(1st \not= d,\; 1st \not= q,\; 2nd \not= d,\; 2nd = q)\\
&\quad + P(1st \not= d,\; 1st = q,\; 2nd \not= d,\; 2nd \not= q)\\
&= \frac{52 - 13 - 3}{52} \cdot \frac{3}{51} + \frac{3}{52} \cdot \frac{51 - 13 - 2}{51} = \boxed{\frac{18}{221}}\\
p_{X, Y}(0, 2) &= P(1st \not= d,\; 1st = q,\; 2nd \not= d,\; 2nd = q)\\
&= \frac{3}{52} \cdot \frac{2}{51} = \boxed{\frac{1}{442}}\\
\end{align*}
\begin{align*}
p_{X, Y}(1, 0) &= P(1st = d,\; 1st \not= q,\; 2nd \not= d,\; 2nd \not= q)\\
&\quad + P(1st \not= d,\; 1st \not= q,\; 2nd = d,\; 2nd \not= q)\\
&= \frac{12}{52} \cdot \frac{51 - 12 - 3}{51} + \frac{52 - 13 - 3}{52} \cdot \frac{12}{51} = \boxed{\frac{72}{221}}\\
p_{X, Y}(1, 1) &= P(1st = d,\; 1st \not= q,\; 2nd \not= d,\; 2nd = q)\\
&\quad + P(1st \not= d,\; 1st = q,\; 2nd = d,\; 2nd \not= q)\\
&\quad + P(1st = d, 1st = q, 2nd \not= d, 2nd \not= q)\\
&\quad + P(1st \not= d, 1st \not= q, 2nd = d, 2nd = q)\\
&= \frac{12}{52} \cdot \frac{3}{51} + \frac{3}{52} \cdot \frac{12}{51} + \frac{1}{52} \cdot \frac{51 - 12 - 3}{51} + \frac{52 - 13 - 3}{52} \cdot \frac{1}{51}\\
&= \boxed{\frac{12}{221}}\\
\end{align*}
\begin{align*}
p_{X, Y}(1, 2) &= P(1st = d,\; 1st = q,\; 2nd \not= d,\; 2nd = q)\\
&\quad + P(1st \not= d,\; 1st = q,\; 2nd = d,\; 2nd = q)\\
&= \frac{1}{52} \cdot \frac{3}{51} + \frac{3}{52} \cdot \frac{1}{51}\\
&= \boxed{\frac{1}{442}}\\
p_{X, Y}(2, 0) &= P(1st = d,\; 1st \not= q,\; 2nd = d,\; 2nd \not= q)\\
&= \frac{12}{52} \cdot \frac{11}{51} = \boxed{\frac{11}{221}}\\
p_{X, Y}(2, 1) &= P(1st = d,\; 1st = q,\; 2nd = d,\; 2nd \not= q)\\
&\quad + P(1st = d,\; 1st \not= q,\; 2nd = d,\; 2nd = q)\\
&= \frac{1}{52} \cdot \frac{12}{51} + \frac{12}{52} \cdot \frac{1}{51}\\
&= \boxed{\frac{2}{221}}\\
p_{X, Y}(2, 2) &= P(1st = d,\; 1st = q,\; 2nd = d,\; 2nd = q)\\
&= \frac{1}{52} \cdot 0\\ 
&= \boxed{0}\\
\end{align*}
\\
\\
\\
2.
\begin{align*}
P_X(0) &= \frac{105}{221} + \frac{18}{221} + \frac{1}{442}\\
&= \boxed{\frac{19}{34}}\\
p_X(1) &= \frac{72}{221} + \frac{12}{221} + \frac{1}{442}\\
&= \boxed{\frac{13}{34}}\\
p_X(2) &= \frac{11}{221} + \frac{2}{221} + 0\\
&= \boxed{\frac{1}{17}}\\
\sum_{i = 0}^3 p_X(i) &= \frac{19}{34} + \frac{13}{34} + \frac{1}{17} = 1\\
\end{align*}
\begin{align*}
p_Y(0) &= \frac{105}{221} + \frac{72}{221} + \frac{11}{221}\\
&= \boxed{\frac{188}{221}}\\
p_Y(1) &= \frac{18}{221} + \frac{12}{221} + \frac{2}{221}\\
&= \boxed{\frac{32}{221}}\\
p_Y(2) &= \frac{1}{442} + \frac{1}{442}\\
&= \boxed{\frac{1}{221}}\\
\sum_{i = 1}^3 p_Y(i) &= \frac{188}{221} + \frac{32}{221} + \frac{1}{221} = 1\\
\end{align*}
\\
\\
\\
\\
3.
\begin{align*}
P(Y \geq 1 \;|\; X \geq 1) &= \frac{P(Y \geq 1,\; X \geq 1)}{P(X \geq 1)}\\
&= \frac{p_{X, Y}(1, 1) + p_{X, Y}(1, 2) + p_{X, Y}(2, 1) + p_{X, Y}(2, 2)}{p_{X}(1) + p_{X}(2)}\\
&= {\large\frac{\frac{12}{221} + \frac{1}{442} + \frac{2}{221} + 0}{\frac{13}{34} + \frac{1}{17}}}\\
&= \frac{29}{195}\\
\end{align*}
\newpage
\noindent
Problem 3\\
1\\
$\mathcal{X}$ := random variable corresponding to number of failures of system $X$\\
$\mathcal{Y}$ := random variable corresponding to number of failures of system $Y$\\
\\
$Y$ has at least two failures per day:
\begin{align*}
P(\mathcal{Y} \geq 2) &= 0.5 + 0.17 + 0.03 = \boxed{0.7}\\
\end{align*}
the number of failures of $X$ is strictly less than 2, and the number of failures of $Y$ is greater than or equal to 3:
\begin{align*}
P(\mathcal{X} < 2,\; \mathcal{Y} \geq 3) &= p_{\mathcal{X}, \mathcal{Y}}(0, 3) + p_{\mathcal{X}, \mathcal{Y}}(0, 4) + p_{\mathcal{X}, \mathcal{Y}}(1, 3) + p_{\mathcal{X}, \mathcal{Y}}(1, 4)\\
&= p_{\mathcal{X}}(0) \cdot p_{\mathcal{Y}}(3) + p_{\mathcal{X}}(0) \cdot p_{\mathcal{Y}}(4) + p_{\mathcal{X}}(1) \cdot p_{\mathcal{Y}}(3) + p_{\mathcal{X}}(1) \cdot p_{\mathcal{Y}}(4)\\
&= (0.07)(0.17) + (0.07)(0.03) + (0.35)(0.17) + (0.35)(0.03)\\
&= \boxed{0.084}\\
\end{align*}
there is only one failure in the day
\begin{align*}
P(\text{number of failure } = 1) &= p_{\mathcal{X}, \mathcal{Y}}(0, 1) + p_{\mathcal{X}, \mathcal{Y}}(1, 0)\\
&= p_{\mathcal{X}}(0) \cdot p_{\mathcal{Y}}(1) + p_{\mathcal{X}}(1) \cdot p_{\mathcal{Y}}(0)\\
&= (0.07)(0.2) + (0.35)(0.1)\\
&= \boxed{0.049}\\
\end{align*}
$X$ has the same number of failures as $Y$
\begin{align*}
P(\mathcal{X} = \mathcal{Y}) &= p_{\mathcal{X}, \mathcal{Y}}(0, 0) + p_{\mathcal{X}, \mathcal{Y}}(1, 1) + p_{\mathcal{X}, \mathcal{Y}})(2, 2) + p_{\mathcal{X}, \mathcal{Y}}(3, 3) + p_{\mathcal{X}, \mathcal{Y}}(4, 4)\\
&= p_{\mathcal{X}}(0) \cdot p_{\mathcal{Y}}(0) + p_{\mathcal{X}}(1) \cdot p_{\mathcal{Y}}(1) + p_{\mathcal{X}}(2) \cdot p_{\mathcal{Y}}(2) + p_{\mathcal{X}}(3) \cdot p_{\mathcal{Y}}(3) + p_{\mathcal{X}}(4) \cdot p_{\mathcal{Y}}(4)\\
&= (0.07)(0.1) + (0.35)(0.2) + (0.34)(0.5) + (0.18)(0.17) + (0.06)(0.03)\\
&= \boxed{0.2794}\\
\end{align*}
\\
\\
\\
\\
2.
\begin{center}
\begin{tabular}{|c|c|c|c|c|c|c|c|}\hline
\diagbox[width = 2cm, height = 1cm]{$\mathcal{X}$}{$\mathcal{Y}$} & 0 & 1 & 2 & 3 & 4 & & $p_X(x)$\\ \hline
0 & $\Centerstack{(0.07)(0.1) \\ = 0.007}$ & $\Centerstack{(0.07)(0.2) \\ = 0.014}$ & $\Centerstack{(0.07)(0.5) \\ = 0.035}$ & $\Centerstack{(0.07)(0.17) \\ = 0.0119}$ & $\Centerstack{(0.07)(0.03) \\ = 0.0021}$ & & 0.07\\ \hline
1 & $\Centerstack{(0.35)(0.1) \\ = 0.035}$ & $\Centerstack{(0.35)(0.2) \\ = 0.07}$ & $\Centerstack{(0.35)(0.5) \\ = 0.175}$ & $\Centerstack{(0.35)(0.17) \\ = 0.0595}$ & $\Centerstack{(0.35)(0.03) \\ = 0.0105}$ & & 0.35\\ \hline
2 & $\Centerstack{(0.34)(0.1) \\ = 0.034}$ & $\Centerstack{(0.34)(0.2) \\ = 0.068}$ & $\Centerstack{(0.34)(0.5) \\ = 0.17}$ & $\Centerstack{(0.34)(0.17) \\ = 0.0578}$ & $\Centerstack{(0.34)(0.03) \\ = 0.0102}$ & & 0.34\\ \hline
3 & $\Centerstack{(0.18)(0.1) \\ = 0.018}$ & $\Centerstack{(0.18)(0.2) \\ = 0.036}$ & $\Centerstack{(0.18)(0.5) \\ = 0.09}$ & $\Centerstack{(0.18)(0.17) \\ = 0.0306}$ & $\Centerstack{(0.18)(0.03) \\ = 0.0054}$ & & 0.18\\ \hline
4 & $\Centerstack{(0.06)(0.1) \\ = 0.006}$ & $\Centerstack{(0.06)(0.2) \\ = 0.012}$ & $\Centerstack{(0.06)(0.5) \\ = 0.03}$ & $\Centerstack{(0.06)(0.17) \\ = 0.0102}$ & $\Centerstack{(0.06)(0.03) \\ = 0.0018}$ & & 0.06\\ \hline
& & & & & & &\\ \hline
$p_Y(y)$ & 0.1 & 0.2 & 0.5 & 0.17 & 0.03 & & 1\\ \hline
\end{tabular}
\end{center}
\begin{align*}
p_{\mathcal{X}}(0) &= 0.007 + 0.014 + 0.035 + 0.0119 + 0.0021\\
&= 0.07\\
p_{\mathcal{X}}(1) &= 0.035 + 0.07 + 0.175 + 0.0595 + 0.0105\\
&= 0.35\\
p_{\mathcal{X}}(2) &= 0.034 + 0.068 + 0.17 + 0.0578 + 0.0102\\
&= 0.34\\
p_{\mathcal{X}}(3) &= 0.018 + 0.036 + 0.09 + 0.0306 + 0.0054\\
&= 0.18\\
p_{\mathcal{X}}(4) &= 0.006 + 0.012 + 0.03 + 0.0102 + 0.0018\\
&= 0.06\\
\sum_{i = 0}^4 p_{\mathcal{X}}(i) &= 0.07 + 0.35 + 0.34 + 0.18 + 0.06\\
&= 1\\
\end{align*}
\begin{align*}
p_{\mathcal{Y}}(0) &= 0.007 + 0.035 + 0.034 + 0.018 + 0.006\\
&= 0.1\\
p_{\mathcal{Y}}(1) &= 0.014 + 0.07 + 0.068 + 0.036 + 0.012\\ 
&= 0.2\\
p_{\mathcal{Y}}(2) &= 0.035 + 0.175 + 0.17 + 0.09 + 0.03\\
&= 0.5\\
p_{\mathcal{Y}}(3) &= 0.0119 + 0.0595 + 0.0578 + 0.0306 + 0.0102\\ 
&= 0.17\\
p_{\mathcal{Y}}(4) &= 0.0021 + 0.0105 + 0.0102 + 0.0054 + 0.0018\\
&= 0.03\\\
\sum_{i = 0}^4 p_{\mathcal{Y}}(i) &= 0.1 + 0.2 + 0.5 + 0.17 + 0.03\\
&= 1\\
\end{align*}
\\
\\
\\
\\
3.
\begin{align*}
E[\mathcal{X}] &= \sum_{i = 0}^4 x_i p_{\mathcal{X}}(x_i)\\
&= 0 \cdot p_{\mathcal{X}}(0) + 1 \cdot p_{\mathcal{X}}(1) + 2 \cdot p_{\mathcal{X}}(2) + 3 \cdot p_{\mathcal{X}}(3) + 4 \cdot p_{\mathcal{X}}(4)\\
&= (0)(0.07) + (1)(0.35) + (2)(0.34) + (3)(0.18) + (4)(0.06)\\
&= \boxed{1.81}\\
E[\mathcal{Y}] &= \sum_{i = 0}^4 y_i p_{\mathcal{Y}}(y_i)\\
&= 0 \cdot p_{\mathcal{Y}}(0) + 1 \cdot p_{\mathcal{Y}}(1) + 2 \cdot p_{\mathcal{Y}}(2) + 3 \cdot p_{\mathcal{Y}}(3) + 4 \cdot p_{\mathcal{Y}}(4)\\
&= (0)(0.1) + (1)(0.2) + (2)(0.5) + (3)(0.17) + (4)(0.03)\\
&= \boxed{1.83}\\
\end{align*}
\newpage
\noindent
Problem 4
{\large
\begin{table}[!ht]
    \centering
        \begin{tabular}{|c|c|c|c|c|c|c|c|} \cline{3-8}
            \multicolumn{2}{c}{} & \multicolumn{5}{|c|}{$x$} & \multicolumn{1}{c|}{\multirow{2}{*}{$p_Y(y)$}}\\ \cline{3-7}
            \multicolumn{2}{c|}{} & 1 & 2 & 3 & 4 & 5 &\\ \hline
            \multirow{5}{*}{$y$} & 1 & 1/14 & 1/14 & 1/14 & 1/14 & 1/14 & 5/14\\ \cline{2-8}
             & 2 & 0 & 1/14 & 1/14 & 1/14 & 1/14 & 4/14\\ \cline{2-8}
             & 3 & 0 & 1/14 & 1/14 & 0 & 0 & 2/14\\ \cline{2-8}
             & 4 & 0 & 1/14 & 1/14 & 0 & 0 & 2/14\\ \cline{2-8}
             & 5 & 0 & 1/14 & 0 & 0 & 0 & 1/14\\ \hline
             \multicolumn{2}{|c|}{$p_X(x)$} & 1/14 & 5/14 & 4/14 & 2/14 & 2/14 & 1\\ \hline
        \end{tabular}
\end{table}
}\\
steps:\\
fill row $y = 1$\\
fill column $x = 2$\\
fill row $y = 5$\\
fill column $x = 1$\\
fill row $y = 2$\\
fill column $x = 3$\\
fill column $x = 4$\\
fill column $x = 5$\\
\newpage
\noindent
Problem 5\\
1.
\begin{align*}
 1 &= \int_{-\infty}^{\infty} \int_{-\infty}^{\infty} f_{X, Y}(x, y)\,dy\,dx\\
&= \int_{0}^{\infty} \int_{0}^{\infty} Ke^{-3x - 2y}\,dy\,dx\\
&= K\int_{0}^{\infty} \int_{0}^{\infty} e^{-3x - 2y}\,dy\,dx\\
&\text{/* } u := -3x - 2y;\quad \partial u = -2 \partial y \text{ */}\\
&= K\int_{0}^{\infty} \frac{1}{-2} \left[e^{-3x - 2y}\right]_{y = 0}^{y = \infty}\,dx\\
&= -\frac{K}{2}\int_{0}^{\infty} \left[e^{-3x - 2y}\right]_{y = 0}^{y = \infty}\,dx\\
&= -\frac{K}{2}\int_{0}^{\infty} (e^{-3x} \cdot 0 - e^{-3x})\,dx\\
&= \frac{K}{2}\int_{0}^{\infty} e^{-3x}\,dx\\
&\text{/* } u_2 := -3x;\quad du_2 = -3dx \text{ */}\\
&= - \frac{K}{6} [e^{-3x}]_0^{\infty}\\
&= - \frac{K}{6} (0  - 1)\\
&= \frac{K}{6}\\
\\
1 &= \frac{K}{6}\\
K &= \boxed{6}\\
\end{align*}
\newpage
\noindent
2.
\begin{align*}
&a := x; \quad b := y\\
F_{X, Y}(a, b) &= P(X \leq a, Y \leq b)\\
&= P(-\infty \leq X \leq a, -\infty \leq Y \leq b)\\
&= \int_{-\infty}^{a} \int_{-\infty}^{b} f_{X, Y}(x, y)\,dy\,dx\\
&= \int_{0}^{a} \int_{0}^{b} 6e^{-3x - 2y}\,dy\,dx\\
&= 6\int_{0}^{a} \int_{0}^{b} e^{-3x - 2y}\,dy\,dx\\
&\text{/* } u := -3x - 2y;\quad \partial u = -2 \partial y \text{ */}\\
&= -3\int_{0}^{a} [e^{-3x - 2y}]_{y = 0}^{y = b}\,dx\\ 
&= -3\int_{0}^{a} (e^{-3x - 2b} - e^{-3x})\,dx\\ 
&= -3\int_{0}^{a} (e^{-3x})(e^{-2b} - 1)\,dx\\
&= -3(e^{-2b} - 1)\int_{0}^{a} e^{-3x}\,dx\\
&\text{/* } u_2 := -3x;\quad du_2 = -3dx \text{ */}\\
&= (e^{-2b} - 1) [e^{-3x}]_0^a\\
&= (e^{-2b} - 1)(e^{-3a} - 1)\\
&= e^{-2b - 3a} - e^{-2b} - e^{-3a} + 1\\
\end{align*}
$\boxed{F_{X, Y}(x, y) =
\begin{cases}
e^{-2y - 3x} - e^{-2y} - e^{-3x} + 1 & \text{if } x > 0,\; y > 0\\
0 & \text{otherwise}\\
\end{cases}}$
\newpage
\noindent
3.
\begin{align*}
P(X \leq 1, Y \leq 2) &= F_{X, Y}(1, 2)\\
&= e^{-2(2) - 3(1)} - e^{-2(2)} + e^{-3(1)} - 1\\
&= \boxed{\frac{1}{e^{7}} - \frac{1}{e^{4}} - \frac{1}{e^{3}} + 1}\\
\end{align*}
\end{document}