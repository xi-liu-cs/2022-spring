\documentclass[12pt,border=4pt,multi]{article} % \documentclass[tikz,border=4pt,multi]{article}
\usepackage{lingmacros}
\usepackage{tree-dvips}
\usepackage{amssymb} % for mathbb{}
\usepackage[dvipsnames]{xcolor}
\usepackage{forest}
\usepackage{amsmath} % for matrices
\usepackage{xeCJK}
\usepackage{tikz}
\usepackage[arrowdel]{physics}
\usepackage{graphicx}
\usepackage{wrapfig}
\usepackage{listings}
\usepackage{pgfplots, pgfplotstable}
\usepackage{diagbox} % diagonal line in cell
\usepackage[usestackEOL]{stackengine}
\usepackage{multirow}
\graphicspath{{./img}} % specify the graphics path to be relative to the main .tex file, denoting the main .tex file directory as ./
\newcommand\Myperm[2][^n]{\prescript{#1\mkern-2.5mu}{}P_{#2}}
\newcommand\Mycomb[2][^n]{\prescript{#1\mkern-0.5mu}{}C_{#2}}
\definecolor{orchid}{rgb}{0.7, 0.4, 1.1}
\begin{document}
\section*{}
1\\
c
\begin{align*}
T(n) &= 2T(\lceil n / 4 \rceil) + n\\
\frac{n}{4^i} &= 1\\
i &= \log_4 n\\
num\_levels &= depth + 1 = \log_4 n + 1\\ 
num\_nodes(depth = i) &= 2^i\\ 
cost\_per\_node(depth = i) &= \frac{n}{4^i}\\
cost\_per\_level(depth = i) &= num\_nodes(depth = i) \cdot cost\_per\_node(depth = i)\\
&= 2^i \cdot \frac{n}{4^i}\\
&= \left(\frac{2}{4}\right)^i n\\
&= \left(\frac{1}{2}\right)^i n\\
&= \frac{n}{2^i}\\
T(n) &= \sum_{i = 0}^{max\_depth} cost\_per\_level(depth = i)\\
&= \sum_{i = 0}^{\log_4 n} \frac{n}{2^i}\\
&= n\sum_{i = 0}^{\log_4 n} \frac{1}{2^i}\\
&= n\left(\frac{1}{2^0} + \frac{1}{2^1} + \frac{1}{2^2} + + ... + \frac{1}{2^{\log_4 n}}\right)\\
&= \Theta(n)\\
\end{align*}
\newpage
\noindent
2\\
c
\begin{lstlisting}[language = c]
/* suppose the elements of a are sorted.
you are given an array a[1...n] 
of length n consisting of some positive
and some negative integers in sorted order.
design a divide and conquer-based algorithm
to find the first positive integer in a. 
your algorithm must work in O(log n) time. 

example: If a = {-7, -3, 6, 8, 11, 12}, 
then the first positive integer listed in a is a[3] = 6, 
so the algorithm must return 6 */

#include <stdio.h>

int find(int * a, int left, int right)
{/* find i s.t. a[i - 1] < 0, a[i] > 0 */
    if(left > right)
        return 1 << 31;
    int i = (left + right) / 2;
    if(!i && a[i] > 0)
        return a[i];
    else if(a[i - 1] < 0 && a[i] > 0)
        return a[i];
    else if(a[i] > 0)
        return find(a, left, i - 1);
    else if(a[i] < 0)
        return find(a, i + 1, right);
}


int main()
{
    int a[] = {-7, -3, 6, 8, 11, 12};
    int n = sizeof(a) / sizeof(*a);
    printf("%d\n", find(a, 0, n - 1));
}
\end{lstlisting}
\leavevmode
\\
\\
\\
\\
\\
3\\
a
\begin{lstlisting}[language = c]
#include <stdio.h>
#include <math.h>

int choose_box_rec(int * a, int n, int k)
{
    if(n - 1 < 0)
        return 0;
    return fmax(choose_box_rec(a, n - 1, k), 
    a[n - 1] + choose_box_rec(a, n - k - 1, k));
}

int choose_box(int * a, int n, int k)
{
    int memo[n];
    for(int i = 0; i < n; ++i)
    {
        int a1 = 0, a2 = 0;
        if(i - 1 >= 0)
            a1 = memo[i - 1];
        if(i - k - 1 >= 0)
            a2 = memo[i - k - 1];
        memo[i] = fmax(a1, a2 + a[i]);   
    }
    return memo[n - 1];
}

int main()
{
    int a[] = {-7, -3, 6, 8, 11, 12};
    int n = sizeof(a) / sizeof(*a);
    int k = 2;
    printf("choose_box_rec = %d\n", choose_box_rec(a, n, k));
    printf("choose_box = %d\n", choose_box(a, n, k));
}
\end{lstlisting}
\end{document}
