\documentclass[12pt,border=4pt,multi]{article}%\documentclass[tikz,border=4pt,multi]{article}
\usepackage{lingmacros}
\usepackage{tree-dvips}
\usepackage{amssymb} %for mathbb{}
\usepackage{xcolor}
\usepackage{forest}
\usepackage{amsmath} %for matrices
\usepackage{xeCJK}
\usepackage{tikz}
\usepackage[arrowdel]{physics}
\usepackage{graphicx}
\usepackage{wrapfig}
\usepackage{listings}
\graphicspath{{./images}} %specify the graphics path to be relative to the main .tex file, denoting the main .tex file directory as ./
\usepackage{esint}
\newcommand\Myperm[2][^n]{\prescript{#1\mkern-2.5mu}{}P_{#2}}
\newcommand\Mycomb[2][^n]{\prescript{#1\mkern-0.5mu}{}C_{#2}}
\definecolor{mypink1}{rgb}{0.75, 0.15, 0.5}

\begin{document}

\section*{Xi Liu, xl3504, Homework 1}
for this entire assignment, I use ``IH" as acronym for ``inductive hypothesis"\\
\\
Question 0:\\
None
\newpage 
\noindent
Question 1:\\
let $P(n)$ be the proposition such that \[(1 - r)(1 + r + r^2 + ... + r^{n - 1}) = 1 - r^n, \qquad \forall n \in \mathbb{N}\]
1.\\
base step:\\ 
$P(1)$ is true, when $n = 1, \quad n - 1 = 0$, substitute 0 for $n - 1$ of left-hand side and substitute 1 for $n$ of right-hand side
\[(1 - r)(1) = 1 - r^1\]
\\
2.\\
inductive step:\\
assume $P(k)$ is true for some positive integer $k$, or equivalently, assume 
\[(1 - r)(1 + r + r^2 + ... + r^{k - 1}) = 1 - r^k\]
is true
\[(1 - r)(1 + r + r^2 + ... + r^{k}) = (1 - r)(1 + r + r^2 + ... + r^{k - 1} + r^k)\]
\begin{align*}
    &= (1 - r)(1 + r + r^2 + ... + r^{k - 1}) + (1 - r)r^k\\
    &\stackrel{\text{IH}}{=} 1 - r^k + (1 - r)r^k\\
    &= 1 - r^k + r^k - r(r^{k})\\
    &= 1 - r^{k + 1}
\end{align*}
so $P(k + 1)$ is true\\
by mathematical induction, $P(n)$ is true $\forall n \in \mathbb{N}$\\
\\
3.\\
rewrite equation (1) in equivalent form: 
\[(1 - r)(1 + r + r^2 + ... + r^{n - 1}) = 1 - r^n\]
\[\sum_{i = 0}^{n - 1}r^i = 1 + r + r^2 + ... + r^{n - 1} = \frac{1 - r^n}{1 - r}\]
\[\sum_{i = 0}^{n}r^i = 1 + r + r^2 + ... + r^{n} = \frac{1 - r^{n + 1}}{1 - r}\]
now evaluate the following sum:
\[2^0 \cdot 3^n + 2^1 \cdot 3^{n - 1} + 2^2 \cdot 3^{n - 2} + ... + 2^n \cdot 3^{n - n} = \sum_{i = 0}^n \left(2^i \cdot 3^{n - i}\right)\]
\begin{align*}
    &= \sum_{i = 0}^n \left(2^i\left(\frac{3^n}{3^i}\right)\right)\\
    &= 3^n \sum_{i = 0}^n \left(\frac{2^i}{3^i}\right)\\
    &= 3^n \sum_{i = 0}^n \left(\frac{2}{3}\right)^i\\
    &\text{apply equation (1) with } r = \frac{2}{3}\\
    &= 3^n\left(\frac{1 - (2 / 3)^{n + 1}}{1 - (2 / 3)} \right)\\
    &= 3^n\left(\frac{1 - (2 / 3)^{n + 1}}{1 / 3}\right)\\
    &= 3 \cdot 3^n\left(1 - (2 / 3)^{n + 1}\right)\\
    &= 3^{n + 1}\left(1 - (2 / 3)^{n + 1}\right)\\
    &= 3^{n + 1} - 2^{n + 1}\\
\end{align*}
\newpage 
\noindent
Question 2:\\
1.\\
false\\ 
let $f(n) = n$, $g(n) = n^2$\\
$f(n) = O(g(n))$ or equivalently $n = O(n^2)$ because there exist positive constants $c$ and $n_0$ such that $0 \leq n \leq c n^2$ for all $n \geq n_0$\\
\[n \leq c n^2\]
dividing by $n$ yields
\[1 \leq c n\]
we can make the inequality hold for any value of $n \geq 1$ by choosing any constant $c \geq 1$\\
\\
claim: but $g(n) \not= O(f(n))$ because $n^2 \not= O(n)$\\
proof: for contradiction, assume $n^2 = O(n)$, then there exist positive constants $c$ and $n_0$ such that $0 \leq n^2 \leq c n$ for all $n \geq n_0$\\
\[n^2 \leq c n\]
dividing by $n$ yields
\[n \leq c\]
which cannot remain true for arbitrary large $n$, since $c$ is a constant\\
\\
\\
\\
\\
2.\\
true\\
$f = O(g)$ means $f(n) \leq c_1 g(n)$ for all $n \geq n_1$, when $c_1$ and $n_1$ are some positive constants\\
$g = O(h)$ means $g(n) \leq c_2 h(n)$ for all $n \geq n_2$, when $c_2$ and $n_2$ are some positive constants\\
when $n \geq n_1$ and $n \geq n_2$, multiply $g(n) \leq c_2 h(n)$ by $c_1$ yields
\[c_1 g(n) \leq c_1 c_2 h(n)\]
so 
\[f(n) \leq c_1 g(n) \leq c_1 c_2 h(n)\]
let $c_3 = c_1 c_2$, 
then 
\[f(n) \leq c_3 h(n)\]
so
\[f(n) = O(h(n))\]
\\
\\
\\
\\
3.\\
false\\
let $f(n) = n^2 + n$, $g(n) = n^2$, then $\forall n \geq 1,\;f(n) > g(n)$\\
$f(n) = O(g(n))$ or equivalently $n^2 + n = O(n^2)$ because there exist positive constants $c$ and $n_0$ such that $0 \leq n^2 + n \leq c n^2$ for all $n \geq n_0$\\
$n^2 + n \leq c n^2$ can remain true when $c \geq 2$ and $n \geq 1$\\
$g(n) = O(f(n))$ or equivalently $n^2 = O(n^2 + n)$ because there exist positive constants $c$ and $n_0$ such that $0 \leq n^2 \leq c (n^2 + n)$ for all $n \geq n_0$\\
$n^2 \leq c (n^2 + n)$ can remain true when $c \geq 1$ and $n \geq 1$
\[f(n) - g(n) = n^2 + n - n^2 = n \not= O(1)\]
proof of $n \not= O(1)$: for contradiction, assume $n = O(1)$, then there exist positive constants $c$ and $n_0$ such that $0 \leq n \leq c \cdot 1$ for all $n \geq n_0$\\
\[n \leq c\]
which cannot remain true for arbitrary large $n$, since $c$ is a constant\\
so $f(n) - g(n) = n \not= O(1)$\\
\\
\\
\\
\\
4.\\
true\\
$f = O(g)$ means $f(n) \leq c_1 g(n)$ for all $n \geq n_1$, when $c_1$ and $n_1$ are some positive constants\\
$g = O(f)$ means $g(n) \leq c_2 f(n)$ for all $n \geq n_2$, when $c_2$ and $n_2$ are some positive constants\\
multiply $g(n) \leq c_2 f(n)$ by $c_1$ yields
\[c_1 g(n) \leq c_1 c_2 f(n)\]
so 
\[f(n) \leq c_1 g(n) \leq c_1 c_2 f(n)\]
divide by $c_1$ yields
\[\frac{f(n)}{c_1} \leq g(n) \leq c_2 f(n)\]
let $f(n)$ be the function that is divided by each of the items in the above inequality, then 
\[\frac{f(n)}{f(n)/c_1} \geq \frac{f(n)}{g(n)} \geq \frac{f(n)}{c_2 f(n)}\]
\[c_1 \geq \frac{f(n)}{g(n)} \geq \frac{1}{c_2}\]
since $f(n) / g(n) \leq c_1 \cdot 1$\\ 
$f / g = O(1)$\\ 
\\
\\
\\
\\
5.\\
false\\
let $h(n) = n$, $f(n) = n^2$, $g(n) = n^3$\\
$f = O(g)$ or equivalently $n^2 = O(n^3)$, since $n^2 \leq c_1 n^3$ for all $n \geq n_1$, when $c_1 \geq 1$ and $n_1 \geq 1$\\
$h = O(g)$ or equivalently $n = O(n^3)$, since $n \leq c_2 n^3$ for all $n \geq n_2$, when $c_2 \geq 1$ and $n_2 \geq 1$\\
but $f \not= O(h)$ or equivalently $n^2 \not= O(n)$, since $n^2 \not\leq c_3 n$ or equivalently $n^2 > c_3 n$ for all $n \geq n_3$, when $c_3 \geq 1$ and $n_3 > c_3$\\
\\
\newpage
\noindent
Question 3:\\
the order is:\\
{\large$\sqrt{n}\log_2 n,\;2^{\log_3 n},\; n^2,\; 2^n,\; n!$}
\newpage
\noindent
Question 4:\\
let $P(n)$ be the proposition that $f_n > 3n$ for all $n > 9$\\
\\
base step:\\
$P(10)$ is true, because $f_{10} = f_9 + f_8 + f_7 = 24 + 13 + 7 = 44 > 3n = 3(10) = 30$\\
$P(11)$ is true, because $f_{11} = f_{10} + f_9 + f_8 = 44 + 24 + 13 = 81 > 3(11) = 33$ \\
$P(12)$ is true, because $f_{12} = f_{11} + f_{10} + f_9 = 81 + 44 + 24 = 149 > 3(12) = 36$\\
\\
inductive step:\\
for the inductive hypothesis, assume $P(k)$ is true for an arbitrary integer $k > 9$; that is, assume that $f_k = f_{k - 1} + f_{k - 2} + f_{k - 3} > 3k$
\[f_{k + 1} = f_{(k + 1) - 1} + f_{(k + 1) - 2} + f_{(k + 1) - 3}
= f_k + f_{k - 1} + f_{k - 2}\]
because of the inductive hypothesis, $f_k > 3k$\\
smallest possible $f_{k - 1}$ is when $k = 10$, then $f_{k - 1} = 
f_{10 - 1} = f_9 = 24$\\
smallest possible $f_{k - 2}$ is when $k = 10$, then $f_{k - 2} = 
f_{10 - 2} = f_8 = 13$\\
so 
\[f_{k + 1} = f_k + f_{k - 1} + f_{k - 2} > 3k + 24 + 13 > 3k + 3 = 3(k + 1)\]
so $P(k + 1)$ is true\\
by mathematical induction $P(n)$ is true for all $n > 9$\\
\newpage
\noindent
Question 5:\\
1.\\
transpositions of array $(7, 5, 2, 6, 9)$:\\
$(1,2),\; (1,3),\; (1,4),\;(2,3)$\\
\\
\\
\\
\\
2.\\
smallest number of transpositions happen when array is sorted in ascending order: $(1, 2, ..., n)$, smallest number of transpositions = 0\\
\\
largest number of transpositions happen when array is sorted in descending order: $(n, n - 1, n - 2, ..., 1)$, there will be $n - i$ transpositions with the $i$th index, then
\[\text{largest number of transpositions} = (n - 1) + (n - 2) + ... + 1\]
\[= \sum_{i = 1}^{n - 1} (n - i) = \sum_{i = 1}^{n - 1} i = \frac{(n - 1)((n - 1) + 1)}{2} = \frac{n(n - 1)}{2}\]
\\
\\
\\
\\
3.\\
\begin{lstlisting}[mathescape = true, escapeinside={;*}{*;)}, language = python]
int merge(int A[], int p, int q, int r)
{
    int left_len = q - p + 1; 
    int right_len = r - q;
    int L[left_len + 1];
    int R[right_len + 1];
    L[left_len + 1] = $\infty$;
    R[right_len + 1] = $\infty$;
    for(int i = 1; i < left_len; i++)
        L[i] = A[p + i];
    for(int i = 1; i < right_len; i++)
        R[i] = A[(q + 1) + i];
    int i = 1;
    int j = 1;
    int counter = 0;
    for(int k = p; k < r; k++)
    {
        if(L[i] <= right[r_i])
        {
            A[k] = L[i];
            i = i + 1;
        }
        else
        {
            A[k] = R[j];
            j = j + 1;
            counter = counter + left_len - i + 1;
        }
    }
    return counter;
}

int merge_transposition(int A[], int p, int r)
{
    int c = 0;
    if(p < r)
    {
        int q = p + $\lfloor$ (r - p) / 2 $\rfloor$;
        int left = merge_transposition(A, p, q);
        int right = merge_transposition(A, q + 1, r);
        c = left + right + merge(A, p, q, r);
    }
    return c;
}
\end{lstlisting}
\newpage
\noindent
correctness of merge\\
\textbf{loop invariant:}\\
at the start of the each for iteration (the loop that rewrites original array $A$ using elements in $L$ and $R$ arrays), $A[p...k - 1]$ contains $final\_index - initial\_index + 1 = (k - 1) - p + 1 = k - p$ smallest elements of $L$ and $R$, in sorted order, $counter$ stores the number of transpositions $(a,\;b)$ such that $p \leq a < p + i - 1$ and $q + 1 \leq b \leq j$\\
Moreover, $L[i]$ and $R[j]$ are smallest elements of their arrays that have not been copied back to $A$\\
\\
\textbf{initialization}:\\ 
before first iteration of the loop, $k = p = 1$, so subarray $A[p...k - 1]$ is empty. This empty subarray contains $k - p = 0$ smallest elements of $L$ and $R$. Since $i = j = 1$, both $L[i]$ and $R[j]$ are the smallest elements of their arrays that have not been copied back to $A$, $i = 1$, $p \leq a < p + 1 - 1$, since $a \geq p$ and $a < p$ is a contradiction, there is no such number $a$ so there is no transposition and $counter$ is 0\\
\\
\textbf{maintenance}: $P(k) \rightarrow P(k + 1)$\\
if $L[i] > R[j]$, then $R[j]$ is the smallest element that has not yet copied back into $A$,\\
based on the inductive hypothesis (loop invariant), $A[p...k - 1]$ contains $k - p$ smallest elements of array $L$ and $R$, and $counter$ stores the number of transpositions $(a,\;b)$ such that $p \leq a < p + i - 1$ and $q + 1 \leq b \leq j$
$\;\;(P(k))$.\\
after line 14 copies $L[i]$ into $A[k]$, the subarray $A[p...k]$ contains $final\_index - initial\_index + 1 = k - p + 1$ smallest elements of array $L$ and $R$. since the assumption is $L[i] > R[j]$, so $R[j]$ is less than left\_len - i + 1 elements of $L[i...left\_len]$, but $i < j$, so there are left\_len - i + 1 transpositions associated with $L[i...left\_len]$ and $R[j]$, so after adding left\_len - i + 1 to $counter$ and increment $j$ by 1 and increment $k$ by 1, the loop invariant is maintained for $k + 1$ $\;\;(P(k + 1))$.\\
if $L[i] \leq R[j]$, then merge appropriate action to maintain the loop invariant with the roles of $L$ and $R$ interchanged\\
\\
\textbf{termination}:\\
at termination, $k = r + 1$, $i = \;$left\_len, $j = \;$right\_len. By the loop invariant, the subarray $A[p...k - 1] = A[p...(r + 1) - 1] = A[p...r]$, contains $final\_index - initial\_index + 1 = r - p + 1$ smallest elements of $L[1...n_1 + 1]$ and $R[1...n_2 + 1]$ in sorted order, $counter$ completed the addition of all transpositions $(a,\;b)$ associated with $L[i...$left\_len] and $R[j]$ for all $p \leq i \leq p \; + \;$left\_len, $q + 1 \leq j \leq q\; + \; $right\_len, $p \leq a < p + i - 1$, and $q + 1 \leq b \leq j$\\
\\
\\
\\
show $\Theta(n\lg n)$ is the run time bounds:\\
recurrence is given by\\
$T(n) =
\begin{cases}
    \Theta(1) & if\;n = 1\\
    T(\lceil n / 2 \rceil) + T(\lfloor n / 2 \rfloor) + \Theta(n) & if\;n > 1\\
\end{cases}
$\\
\\
assume $T(n) \leq c(n - 2)\lg(n - 2)$ is true for all positive $m < n$, in particular for $m = \lceil n / 2 \rceil$ and $m = \lfloor n / 2 \rfloor$, or equivalently assume $T(\lceil n / 2 \rceil) \leq c(\lceil n / 2 \rceil - 2)\lg(\lceil n / 2 \rceil - 2)$ and $T(\lfloor n / 2 \rfloor) \leq c(\lfloor n / 2 \rfloor - 2)\lg(\lfloor n / 2 \rfloor - 2)$\\
assume $\Theta(n) = kn$ for the recurrence stated above, where $k$ is a positive constant\\
\begin{align*}
    T(n) &= \textcolor{orchid}{T(\lceil n / 2 \rceil)} + \textcolor{orchid}{T(\lfloor n / 2 \rfloor)} + kn\\
    &\leq \textcolor{orchid}{c(\lceil n / 2 \rceil - 2)\lg(\lceil n / 2 \rceil - 2)} + \textcolor{orchid}{c(\lfloor n / 2 \rfloor - 2)\lg(\lfloor n / 2 \rfloor - 2)} + kn\\
    &\leq c(n / 2 + 1 - 2)\lg(n / 2 + 1 - 2) + c(n / 2 + 1 - 2)\lg(n / 2 + 1 - 2) + kn\\
    &= 2c(n / 2 + 1 - 2)\lg(n / 2 + 1 - 2) + kn\\
    &= 2c(n / 2 - 1)\lg(n / 2 - 1) + kn\\
    &= 2c((n - 2)/ 2)\lg((n - 2)/ 2) + kn\\
    &= c(n - 2)\lg((n - 2)/ 2) + kn\\
    &= c(n - 2)(\lg(n - 2) - \lg 2) + kn\\
    &= c(n - 2)(\lg(n - 2) - 1) + kn\\
    &= c(n - 2)\lg(n - 2) - c(n - 2) + kn\\
    &= c(n - 2)\lg(n - 2) + n(k - c) + 2c\\
    &= c(n - 2)\lg(n - 2) - (n(c - k) - 2c)\\
    &\leq c(n - 2)\lg(n - 2) \qquad if\;n(c - k) - 2c \geq 0\\
\end{align*}
\[n(c - k) - 2c \geq 0\]
\[n(c - k) \geq 2c\]
\[n \geq \frac{2c}{c - k}\]
\[\text{so, pick} \quad n \geq n_0 = \frac{2c}{c - k} \quad \text{and} \quad c > k\]
so
\[T(n) = O(n \lg n)\]
\\
\\
\\
\\
need to show $T(n) \geq cn\lg n$ for all $n \geq n_0$, where $c$ and $n_0$ are positive constants\\
\\
assume $T(n) \geq c(n + 2)\lg(n + 2)$ is true for all positive $m < n$, in particular for $m = \lceil n / 2 \rceil$ and $m = \lfloor n / 2 \rfloor$, or equivalently assume $T(\lceil n / 2 \rceil) \geq c(\lceil n / 2 \rceil + 2)\lg(\lceil n / 2 \rceil + 2)$ and $T(\lfloor n / 2 \rfloor) \geq c(\lfloor n / 2 \rfloor + 2)\lg(\lfloor n / 2 \rfloor + 2)$
\begin{align*}
    T(n) &= \textcolor{orchid}{T(\lceil n / 2 \rceil)} + \textcolor{orchid}{T(\lfloor n / 2 \rfloor)} + kn\\
    &\geq c(\lceil n / 2 \rceil + 2)\lg(\lceil n / 2 \rceil + 2) + c(\lfloor n / 2 \rfloor + 2)\lg(\lfloor n / 2 \rfloor + 2) + kn\\
    &\geq c(n / 2 - 1 + 2)\lg(n / 2 - 1 + 2) + c(n / 2 - 1 + 2)\lg(n / 2 - 1 + 2) + kn\\
    &\geq 2c(n / 2 - 1 + 2)\lg(n / 2 - 1 + 2) + kn\\
    &= 2c(n / 2 + 1)\lg(n / 2 + 1) + kn\\
    &= 2c((n + 2) / 2)\lg((n + 2) / 2) + kn\\
    &= c(n + 2)(\lg(n + 2) - \lg 2) + kn\\
    &= c(n + 2)\lg(n + 2) - c(n + 2) + kn\\
    &= c(n + 2)\lg(n + 2) + n(k - c) - 2c\\
    &\geq c(n + 2)\lg(n + 2)\\
    &if\;n(k - c) - 2c \geq 0, \quad
    i.e. \quad n \geq \frac{2c}{k - c}\\
\end{align*}
\[\text{so, pick} \quad n \geq n_0 = \frac{2c}{k - c} \quad \text{and} \quad k > c\]
so 
\[T(n) = \Omega(n \lg n)\]
because $T(n) = O(n \lg n)$ and $T(n) = \Omega(n \lg n)$, so $T(n) = \Theta(n \lg n)$
\newpage
\noindent
Question 6:\\
the inductive step is false for $n = 2$\\
if the set $S$ has 2 horses and $S = \{A,\; B\}$,\\ if exclude $A$, and look at the set $\{B\}$, all of the remaining element(s) have the same color because there is only 1 element in $\{B\}$\\
if exclude $B$, and look at the set $\{A\}$, all of the remaining element(s) have the same color because there is only 1 element in $\{A\}$\\
but this does not show that $A$ and $B$ must have the same color\\
\newpage
\noindent
Honors Question 1:\\
\[\tilde{O}(g(n))
= \{f(n): \text{there exist positive constants $c$ and $n_0$ such that}\]
\[0 \leq f(n) \leq \log_2 n \cdot g(n),\;\forall n \geq n_0\}\]
this new notion have transitivity: 
\[f(n) = \tilde{O}(g(n)) \text{ and } g(n) = \tilde{O}(h(n)) \text{ imply } f(n) = \tilde{O}(h(n))\]
\newpage
\noindent
Honors Question 2:\\
let $P(n)$ be the proposition that a convex $n$-gon has $n(n - 3)/2$ diagonals\\
\\
base step:\\
$P(3)$ is true, because the convex 3-gon is a triangle, it has $3(3 - 3) / 2 = 0$ diagonals\\
\\
inductive step:\\
assume $P(k)$ is true, or equivalently assume a convex $k$-gon has $k(k - 3)/2$ diagonals\\
if 1 vertex is added to the convex $k-$gon, then $(k + 1) - 2$ new lines can be drawn, so the number of diagonals in the newly formed convex $k + 1$-gon is 
\[\text{number of diagonals in convex $k$-gon} + (k + 1) - 2\]
\begin{align*}
    &= k(k - 3) / 2 + (k + 1) - 2\\
    &= (k^2 - 3k) / 2 + k - 1\\
    &= k^2 / 2 - 3k / 2 + k - 1\\
    &= k^2 / 2 - k / 2 - 1\\
    &= \frac{1}{2}(k^2 - k - 2)\\
    &= \frac{1}{2}(k - 2)(k + 1)\\
    &= (k + 1)((k + 1) - 3) / 2\\
\end{align*}
so $P(k + 1)$ is true\\
by mathematical induction $P(n)$ is true for all $n \geq 3$\\
\end{document}