\documentclass[12pt,border=4pt,multi]{article}%\documentclass[tikz,border=4pt,multi]{article}
\usepackage{lingmacros}
\usepackage{tree-dvips}
\usepackage{amssymb} %for mathbb{}
\usepackage[dvipsnames]{xcolor}
\usepackage{forest}
\usepackage{amsmath} %for matrices
\usepackage{xeCJK}
\usepackage{tikz}
\usepackage[arrowdel]{physics}
\usepackage{graphicx}
\usepackage{wrapfig}
\usepackage{listings}
\graphicspath{{./img}} %specify the graphics path to be relative to the main .tex file, denoting the main .tex file directory as ./
\usepackage{esint}
\newcommand\Myperm[2][^n]{\prescript{#1\mkern-2.5mu}{}P_{#2}}
\newcommand\Mycomb[2][^n]{\prescript{#1\mkern-0.5mu}{}C_{#2}}
\definecolor{orchid}{rgb}{0.7, 0.4, 1.1}

\begin{document}

\section*{Xi Liu, xl3504, Homework 1}
for this entire assignment, I use ``IH" as acronym for ``inductive hypothesis"\\
\\
Problem 1\\
let $p(n)$ be the proposition that 
\[1 \times 1! + 2 \times 2! + ... + n \times n! = (n + 1)! - 1 \qquad \forall n \in \mathbb{N}\]
1.\\ 
base step:\\
$p(1)$ is true, when $n = 1, \; 1 \times 1! = 1 = (1 + 1)! - 1 = 2 - 1 = 1$\\
\\
2.\\
inductive step:\\
assume $p(k)$ is true for some positive integer $k$, or equivalently, assume 
\[1 \times 1! + 2 \times 2! + ... + k \times k! = (k + 1)! - 1\]
is true
\[1 \times 1! + 2 \times 2! + ... + k \times k! + (k + 1) \times (k + 1)!\]
\begin{align*}
    &\stackrel{\text{IH}}{=} ((k + 1)! - 1) + (k + 1) \times (k + 1)!\\
    &= (k + 1)! - 1 + (k + 1) \times (k + 1)!\\
    &= (k + 2)(k + 1)! - 1\\
    &= (k + 2)! - 1\\
    &= ((k + 1) + 1)! - 1\\
\end{align*}
so $p(k + 1)$ is true\\
by mathematical induction, $p(n)$ is true $\forall n \in \mathbb{N}^+$\\
\newpage 
\noindent
Problem 2\\
let $p(n)$ be the proposition that 
\[a_n = 2^n + 1\]
1.\\
base step:\\
$p(1)$ is true, when $n = 1, a_1 = 2^1 + 1 = 3$\\
$p(2)$ is true, when $n = 2, a_2 = 2^2 + 1 = 5$\\
\\
\\
\\
2.\\
inductive step:\\
assume $p(i)$ is true for some positive integers $i,\;k$ such that $1 \leq i \leq k$, in particular for $i = k$ and $i = k - 1$:\\
\[a_k = 2^k + 1\]
\[a_{k - 1} = 2^{k - 1} + 1\]
is true
\begin{align*}
    a_{k + 1} &= 3a_k - 2a_{k - 1}\\
    &= 3(2^k + 1) - 2(2^{k - 1} + 1)\\
    &= 3 \cdot 2^k + 3 - 2^k - 2\\
    &= (3 - 1)\cdot 2^k + 3 - 2\\
    &= 2\cdot 2^k + 1\\
    &= 2^{k + 1} + 1\\
\end{align*}
so $p(k + 1)$ is true\\
by mathematical induction, $p(n)$ is true $\forall n \in \mathbb{N}^+$\\
\newpage
\noindent
Problem 3\\
the order is:\\
{\large 
(e)\;$2^{\log_2 n},\;
(b)\;n^{1.5} \log_2 n,\;
(c)\;n^2 - 1,\;
(a)\;2^n,\;
(f)\;3^n,\;
(d)\;n!
$}
\newpage
\noindent
Problem 4\\
a.\\
false\\ 
let $f(n) = n$, $g(n) = n^2$\\
$f(n) = O(g(n))$ or equivalently $n = O(n^2)$ because there exist positive constants $c$ and $n_0$ such that $0 \leq n \leq c n^2$ for all $n \geq n_0$
\[n \leq c n^2\]
dividing by $n$ yields
\[1 \leq c n\]
we can make the inequality hold for any value of $n = n_0 \geq 1$ by choosing any constant $c \geq 1$\\
\\
claim: but $g(n) \not= O(f(n))$ because $n^2 \not= O(n)$\\
proof: for contradiction, assume $n^2 = O(n)$, then there exist positive constants $c$ and $n_0$ such that $0 \leq n^2 \leq c n$ for all $n \geq n_0$\\
\[n^2 \leq c n\]
dividing by $n$ yields
\[n \leq c\]
which cannot remain true for arbitrary large $n$, since $c$ is a constant\\
\\
\\
\\
\\
b.\\ 
true\\
$f = \Omega(h)$ means $f(n) \geq c_1 h(n)$ for all $n \geq n_1$, when $c_1$ and $n_1$ are some positive constants\\
$g = \Omega(h)$ means $g(n) \geq c_2 h(n)$ for all $n \geq n_2$, when $c_1$ and $n_1$ are some positive constants
\[(f + g)(n) = f(n) + g(n) \geq c_1 h(n) + c_2 h(n) = (c_1 + c_2)h(n)\]
when $n \geq n_1$ and $n \geq n_2$, let $c_3 = c_1 + c_2$\\
since 
\[(f + g)(n) \geq c_3 h(n)\]
\[(f + g)(n) =  \Omega(h(n))\]
\\
\\
\\
\\
c.\\
true\\
$f = O(g)$ means $f(n) \leq c_1 g(n)$ for all $n \geq n_1$, when $c_1$ and $n_1$ are some positive constants\\
$g = O(h)$ means $g(n) \leq c_2 h(n)$ for all $n \geq n_2$, when $c_2$ and $n_2$ are some positive constants\\
when $n \geq n_1$ and $n \geq n_2$, multiply $g(n) \leq c_2 h(n)$ by $c_1$ yields
\[c_1 g(n) \leq c_1 c_2 h(n)\]
so 
\[f(n) \leq c_1 g(n) \leq c_1 c_2 h(n)\]
let $c_3 = c_1 c_2$, 
then 
\[f(n) \leq c_3 h(n)\]
so
\[f(n) = O(h(n))\]
\\
\\
\\
\\
d.\\
false\\
let $g(n) = n$, $f(n) = n^2$, $h(n) = n^3$\\
$f = \Omega(g)$ or equivalently $n^2 = \Omega(n)$, since $n^2 \geq c_1 n$ for all $n \geq n_1$, when $c_1 \geq 1$ and $n_1 \geq 1$\\
$h = \Omega(g)$ or equivalently $n^3 = \Omega(n)$, since $n^3 \geq c_2 n$ for all $n \geq n_2$, when $c_2 \geq 1$ and $n_2 \geq 1$\\
but $f \not= \Omega(h)$ or equivalently $n^2 \not= \Omega(n^3)$, since $n^2 \not\geq c_3 n^3$ or equivalently $n^2 < c_3 n^3$ for all $n \geq n_3$, when $c_3 \geq 1$ and $n_3 > c_3$\\
\newpage
\noindent
Problem 5\\
(a)
\begin{lstlisting}
SELECTION_SORT(A)
    for i = A.length downto 2
        max_i = i;
        for j = i - 1 downto 1
            if(A[j] > A[max_i])
                max_i = j
        max = A[max_i]
        for k = max_i + 1 to i
            A[k - 1] = A[k]
        A[i] = max
\end{lstlisting}
\\
\\
\\
\\
(b)\\
best case: $\Theta(n^2)$\\
worst case: $\Theta(n^2)$\\
since there are 2 levels of nested for loops and the lines within the innermost level take constant time
\\
\\
\\
\\
(c)\\
the variant of SELECTION\_SORT that finds the 2 largest elements among the remaining elements and place them at the end is not more time-efficient to the original SELECTION\_SORT\\
for this variant of SELECTION\_SORT: from the code below, although the number of outer for loop is approximately half of the number of original for loop (since $i$ is subtracted by 2 in each iteration of the for loop), but inner for loop created by FIND\_2MAX have more assignment operations and comparisons\\
because for the variant of SELECTION\_SORT, there are also 2 levels of nested for loops and the lines within the innermost level take constant time, so the variant of SELECTION\_SORT that finds the 2 largest elements have complexity:\\
best case: $\Theta(n^2)$\\
worst case: $\Theta(n^2)$\\
\begin{lstlisting}
SWAP(A, i, j):
    t = A[i]
    A[i] = A[j]
    A[j] = t
    
FIND_2MAX(A)
    fst = 0
    sec = -1
    for i = 1 to A.length
        if A[i] > A[fst]
            sec = fst
            fst = i
        else if A[i] < A[fst]
            if sec == - 1 or A[i] > a[sec]
                sec = i
        else if A[i] == A[fst] and i != fst
            sec = i 
    return fst, sec
    
SELECTION_SORT(A):
    for i = A.length downto 1 by -2
        fst, sec = FIND_2MAX(A, i + 1)
        SWAP(A, i, fst)
        if sec != i
            SWAP(A, i - 1, sec)
\end{lstlisting}
\newpage
\noindent
Problem 6\\
COMPARE-1 returns TRUE when every sum of paired elements $A[i] + C[i]$ is less than $B[j]$ for all $1 \leq i \leq n, \; 1 \leq j \leq n$\\
\\
for example, if the function is called with the below input, then it will return TRUE
\begin{lstlisting}
A = {0, 1, 2}
C = {0, 1, 2}
B = {5, 6, 7}
n = 3
\end{lstlisting}
since
\begin{lstlisting}
A[1] + C[1] = 0 + 0 = 0 < B[1] = 5
A[1] + C[1] = 0 + 0 = 0 < B[2] = 6
A[1] + C[1] = 0 + 0 = 0 < B[3] = 7
A[2] + C[2] = 1 + 1 = 2 < B[1] = 5
A[2] + C[2] = 1 + 1 = 2 < B[2] = 6
A[2] + C[2] = 1 + 1 = 2 < B[3] = 7
A[3] + C[3] = 2 + 2 = 4 < B[1] = 5
A[3] + C[3] = 2 + 2 = 4 < B[2] = 6
A[3] + C[3] = 2 + 2 = 4 < B[3] = 7
\end{lstlisting}
\leavevmode
\\
\\
\\
\\
COMPARE-2 returns TRUE when the maximum value of $A[i] + C[i]$ is less then $B[j]$ for all $1 \leq i \leq n, \; 1 \leq j \leq n$\\
\\
so COMPARE-2 will return the same boolean value as COMPARE-1 if given the same input arrays\\
using the same input as above, if the function is called with the below input, then it will return TRUE
\begin{lstlisting}
A = {0, 1, 2}
C = {0, 1, 2}
B = {5, 6, 7}
n = 3
\end{lstlisting}
since
\begin{lstlisting}[mathescape = true]
$aux$ = A[3] + C[3] = 2 + 2 = 4 < B[1] = 5
$aux$ = A[3] + C[3] = 2 + 2 = 4 < B[2] = 6
$aux$ = A[3] + C[3] = 2 + 2 = 4 < B[3] = 7
\end{lstlisting}
\leavevmode
\\
\\
\\
\\
(b)\\
COMPARE-1 worst case running time: $\Theta(n^2)$\\
since there are 2 levels of nested for loops and the lines within the innermost level take constant time\\
\\
COMPARE-2 worst case running time: $\Theta(n)$\\
since there is 1 level of for loops and the lines within the for loop take constant time\\
\end{document}