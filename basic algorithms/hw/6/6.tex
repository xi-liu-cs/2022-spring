\documentclass[12pt,border=4pt,multi]{article}%\documentclass[tikz,border=4pt,multi]{article}
\usepackage{lingmacros}
\usepackage{tree-dvips}
\usepackage{amssymb} %for mathbb{}
\usepackage[dvipsnames]{xcolor}
\usepackage{forest}
\usepackage{amsmath} %for matrices
\usepackage{xeCJK}
\usepackage{tikz}
\usepackage[arrowdel]{physics}
\usepackage{graphicx}
\usepackage{wrapfig}
\usepackage{listings}
\usepackage{pgfplots, pgfplotstable}
\graphicspath{{./img}} %specify the graphics path to be relative to the main .tex file, denoting the main .tex file directory as ./
\newcommand\Myperm[2][^n]{\prescript{#1\mkern-2.5mu}{}P_{#2}}
\newcommand\Mycomb[2][^n]{\prescript{#1\mkern-0.5mu}{}C_{#2}}
\definecolor{orchid}{rgb}{0.7, 0.4, 1.1}

\begin{document}

\section*{Xi Liu, xl3504, Homework 6}
Problem 1\\
assume $T(n) \geq c2^{n/2}$ for all positive $m < n$, $c \in \mathbb{R^+}$, in particular for $m = n - 2$, the inductive hypothesis is $\textcolor{orchid}{T(n - 2) \geq c2^{(n - 2)/4}}$\\
\begin{align*}
T(n) &= T(n - 1) + T(n - 2) + c\\
&= T(n - 1) + T(n - 2) + c\\
&\text{/* since } T(n - 1) \geq T(n - 2) \text{ */}\\
&\geq T(n - 2) + T(n - 2) + c\\
&= 2\textcolor{orchid}{T(n - 2)} + c\\
&\geq 2(\textcolor{orchid}{c2^{(n - 2)/4}}) + c\\
&= \frac{2}{2^4} c2^{n - 2} + c\\
&= \frac{2}{2^6}c2^n + c\\
&= \frac{c}{2^5}2^n + c\\
\end{align*}
$T(n) \geq \frac{c}{2^5}2^n + c$, so $T(n) = \Omega(2^n) = \Omega(2^{n/2})$\\
\newpage
\noindent
Problem 2\\
(a)
\begin{lstlisting}[language = c++]
/* to find the recursion,
the recursive version is included below */

#include <iostream>
#include <string.h>
using namespace std;

int lcs_rec(const char * a, const char * b, int i, int j, int cnt)
{
    if(!i || !j)
        return cnt;
    if(a[i - 1] == b[j - 1])
        cnt = lcs_rec(a, b, i - 1, j - 1, cnt + 1);
    cnt = max(cnt, 
              max(lcs_rec(a, b, i - 1, j, 0),
              lcs_rec(a, b, i, j - 1, 0))
              );
    return cnt;
}
int main()
{
    const char * a = ...
    const char * b = ...
    cout << lcs_rec(a, b, strlen(a), strlen(b), 0);
}
\end{lstlisting}
so the recursion is\\
$\boxed{lcs(i, j, cnt) 
= \begin{cases}
cnt & \text{if } i = 0 \;or\; j = 0\\
lcs(i - 1, j - 1, cnt + 1) & \text{if } a[i - 1] = b[j - 1]\\
max(cnt, lcs(i - 1, j, 0), lcs(i, j - 1, 0)) & \text{if } a[i - 1] \not= b[j - 1]\\
\end{cases}}$
\newpage
\noindent
(b)\\
if $n = strlen(S) = 0$ or $m = strlen(T) = 0$, then $lcs(i, j, cnt) = cnt = 0$\\
\\
\\
(c)
\begin{lstlisting}[language = c, mathescape = true, showstringspaces=false]
#include <stdio.h>
#include <string.h>
#include <math.h>

int lcs(char * a, int m, char * b, int n)
{
    int memo[m + 1][n + 1];
    int ret = 0;
    for(int i = 0; i <= m; ++i)
    {
        for(int j = 0; j <= n; ++j)
        {
            if(!i || !j)
                memo[i][j] = 0;
            else if(a[i - 1] == b[j - 1])
            {
                memo[i][j] = memo[i - 1][j - 1] + 1;
                ret = fmax(ret, memo[i][j]);
            }
            else
                memo[i][j] = 0;
        }
    }
    return ret;
}

int main()
{
   char * a = ...
   char * b = ...
   printf($``\%d$\n", lcs(a, strlen(a), b, strlen(b)));
}
\end{lstlisting}
\\
\\
\\
(d)\\
time complexity = $\Theta(mn)$\\
2 levels of for loop\\
operations within the innermost level are constant\\
\\
\\
\\
\\
Problem 3\\
\begin{lstlisting}[language = c]
#include <stdio.h>
#include <stdlib.h>
#include <string.h>
#include <math.h>

int coin(int *** m)
{
    int ** memo = malloc(n * sizeof(int *));
    *m = memo;
    for(int i = 0; i < n; ++i)
    {
        size_t sz = n * sizeof(int);
        memo[i] = malloc(sz);
        memset(memo[i], 0, sz);
    }
    for(int diag = 0; diag < n; ++diag)
    {
        for(int i = 0, j = diag; j < n; ++i, ++j)
        {/* only use values from upper right triangular matrix */
            int a1 = (i + 2) <= j ? memo[i + 2][j] : 0;
            int a2 = ((i + 1) <= (j - 1)) ? memo[i + 1][j - 1] : 0;
            int a3 = (i <= (j - 2)) ? memo[i][j - 2] : 0;
            memo[i][j] = fmax(a[i] + fmin(a1, a2),
                              a[j] + fmin(a2, a3));
        }
    }
    return memo[0][n - 1];
}

int main()
{
   int ** m;
   printf("coin = %d\n", coin(&m));
}
\end{lstlisting}
\leavevmode
\\
\\
\\
time complexity = $\Theta(n^2)$\\
\newpage
\noindent
Problem 4\\
(a)\\
$P(i, j) = 
\begin{cases}
1 & \text{ if } i = j\\
P(i + 1, j - 1) + 2 & \text{ if } S[i] = S[j]\\
max(P(i + 1, j), P(i, j - 1)) & \text{ if } S[i] \not= S[j]\\ 
\end{cases}$
\\
\\
\\
\\
(b)
base cases:
a subsequence of 1 character is a palindrome of length 1\\
so if $i = j,\quad P(i, j) = 1$\\
\\
\\
\\
(c)
\begin{lstlisting}[language = c]
#include <stdlib.h>
#include <string.h>
#include <math.h>

int longest_palin_subseq(char * a)
{/* consecutive not required */
    int n = strlen(a);
    int memo[n][n];
    memset(memo, 0, sizeof(memo));
    for(int i = 0; i < n; ++i)
        memo[i][i] = 1;
    for(int sub_len = 2; sub_len <= n; ++sub_len)
    {
        for(int i = 0; i < n - sub_len + 1; ++i)
        {
            int j = i + sub_len - 1;
            if(a[i] == a[j])
                memo[i][j] = memo[i + 1][j - 1] + 2;
            else
                memo[i][j] = fmax(memo[i + 1][j], memo[i][j - 1]);
        }
    }
    return memo[0][n - 1];
}
\end{lstlisting}
\leavevmode
\\
\\
\\
(d)\\
time complexity = $\Theta(n^2)$\\
2 levels of for loop\\
operations within the innermost level are constant\\
\newpage
\noindent
Problem 5
\begin{lstlisting}[language = c]
/* longest_palin_subseq2() computes 
the length of the longest palindromic subsequence
using longest_common_subseq() */

#include <stdio.h>
#include <stdlib.h>
#include <string.h>
#include <math.h>

int longest_common_subseq(char * a, char * b)
{
    int a_len = strlen(a), b_len = strlen(b);
    int c[a_len + 1][b_len + 1];
    for(int i = 0; i <= a_len; ++i)
    {
        for(int j = 0; j <= b_len; ++j)
        {
            if(!i || !j)
                c[i][j] = 0;
            else if(a[i - 1] == b[j - 1])
                c[i][j] = c[i - 1][j - 1] + 1;
            else
                c[i][j] = fmax(c[i - 1][j], c[i][j - 1]);
        }
    }
    return c[a_len][b_len];
}

char * revstr(char * a)
{
    int a_len = strlen(a);
    char * rev = malloc((a_len + 1) * sizeof(char));
    strcpy(rev, a);
    for(int i = 0, j = a_len - 1; i < j; ++i, --j)
    {
        char t = rev[i];
        rev[i] = rev[j];
        rev[j] = t;
    }
    return rev;
}

int longest_palin_subseq2(char * a)
{
    char * rev = revstr(a);
    return longest_common_subseq(a, rev);
}
\end{lstlisting}
\newpage
\noindent
Problem 6\\
(a)\\
$max\_reward(i, j) =\\
\begin{cases}
A[0][0] & \text{if } i = 0 \wedge j = 0\\
A[i][j] + max(max\_reward(i - 1, j), max\_reward(i, j + 1)) & \text{else}\\
\end{cases}$
\\
\\
\\
\\
(b)\\
at the starting position $i = 0 \wedge j = 0$, only possible gold to be collected is $A[0][0]$\\
if $i = 0 \wedge j = 0,\;\; max\_reward(i, j) = A[0][0]$\\ 
\\
\\
\\
\\
(c)
\begin{lstlisting}[language = c]
int max_reward()
{ /* each entry in a[][] store the reward */
    int memo[m][n];
    memset(memo, 0, sizeof(memo));
    **memo = **a;
    for(int i = 0; i < m; ++i)
    {
        for(int j = 0; j < n; ++j)
        {
            if(!i && !j)
                continue;
            int a1 = (i - 1) >= 0 ? memo[i - 1][j] : 0;
            int a2 = (j - 1) >= 0 ? memo[i][j - 1] : 0;
            memo[i][j] = a[i][j] + fmax(a1, a2);
        }
    }
    return memo[m - 1][n - 1];
}
\end{lstlisting}
\leavevmode
\\
\\
\\
(d)\\
time complexity = $\Theta(mn)$\\
2 levels of for loop\\
operations within the innermost level are constant\\
\\
\\
\\
\\
Problem 7
\begin{lstlisting}[language = c]
#include <stdio.h>
#include <math.h>

int longest_inc_subseq(int * a, int n)
{
    int memo[n];
    for(int i = 0; i < n; ++i)
    {
        memo[i] = 1;
        for(int j = 0; j < i; ++j)
        {
            if(a[j] < a[i])
                memo[i] = fmax(memo[i], memo[j] + 1);
        }
    }
    return memo[n - 1];
}
\end{lstlisting}
time complexity = $\Theta(n^2)$\\
\end{document}