\documentclass[12pt,border=4pt,multi]{article}%\documentclass[tikz,border=4pt,multi]{article}
\usepackage{lingmacros}
\usepackage{tree-dvips}
\usepackage{amssymb} %for mathbb{}
\usepackage[dvipsnames]{xcolor}
\usepackage{forest}
\usepackage{amsmath} %for matrices
\usepackage{xeCJK}
\usepackage{tikz}
\usepackage[arrowdel]{physics}
\usepackage{graphicx}
\usepackage{wrapfig}
\usepackage{listings}
\usepackage{pgfplots, pgfplotstable}
\graphicspath{{./img}} %specify the graphics path to be relative to the main .tex file, denoting the main .tex file directory as ./
\newcommand\Myperm[2][^n]{\prescript{#1\mkern-2.5mu}{}P_{#2}}
\newcommand\Mycomb[2][^n]{\prescript{#1\mkern-0.5mu}{}C_{#2}}
\definecolor{orchid}{rgb}{0.7, 0.4, 1.1}

\begin{document}

\section*{Xi Liu, xl3504, Homework 5}
Problem 1
\begin{lstlisting}[language = c]
/* f() is an algorithm that finds the
nth fibonacci number with O(1) extra space
base cases: f(0) = 0; f(1) = 1 */

int f(int n)
{
    int left = 0, right = 1;
    for(int i = 1; i <= n; i++)
    {
        int t = right;
        right = left + right;
        left = t;
    }
    return left;
}
\end{lstlisting}
\newpage
\noindent
Problem 2\\
(a)
\begin{align*}
maxprice(n) &= \max(P[1] + maxprice(n - 1) - 1,\\
&P[2] + maxprice(n - 2) - 1,\\
&...,\\
&P[n] + maxprice(0))\\
\\
&= \max\biggr(\max_{1 \leq i \leq n - 1}(P[i] + maxprice(n - i) - 1),\\
&\;\; P[n] + maxprice(0)\biggr)\\
\end{align*}
\\
the last argument, $P[n] + maxprice(0)$, in max() function corresponds to having zero cuts and directly selling the rod of length $n$, since there is no cuts, there is no cost of cutting for the last argument. $\max_{1 \leq i \leq n - 1}(P[i] + maxprice(n - i) - 1)$ are the other $n - 1$ arguments in max() that correspond to the maximum price obtained by making a cut of the rod into a left-hand piece of length $i$ and a right-hand remainder of length $n - i$, $\forall i \in [1, n - 1] \cap \mathbb{N}$, and only the right-hand remainder can be cut further\\
all of the $n - 1$ arguments in $\max_{1 \leq i \leq n - 1}(P[i] + maxprice(n - i) - 1)$ involves an extra cut and a corresponding cut cost of \$1 in addition to the cuts already made in the $maxprice(n - i)$ part\\
\\
\\
\newpage
\noindent
(b)\\
base case:
\begin{align*}
maxprice(0) &= 0\\
&\text{/* when the rod has a length of 0 inch,}\\
&\text{the price is \$0, and there is no cut */}\\
maxprice(1) &= P[1]\\
&\text{/* when the rod has a length of 1 inch,}\\
&\text{the price is P[1], and there is no cut */}\\
\end{align*}
\\
(c)
\begin{lstlisting}[language = c]
/* maxprice() returns the maximum selling price
we can get among all possible options, 
with a payment of $1 per cut */

int max(int a, int b)
{  return (a > b) ? a : b;  }

int maxprice(int * P, int n)
{
    int memo[n + 1];
    *memo = 0; 
    for(int i = 1; i <= n; i++)
    {
        memo[i] = P[i];
        for(int j = 1; j <= i; j++)
        {
            memo[i] = max(memo[i], P[j] + memo[i - j] - 1);
        }
    }
    return memo[n];
}
\end{lstlisting}
\leavevmode
\\
\\
\\
let $T(n)$ be the running time of maxprice()
\[T(n) = \Theta(n^2)\]
since there are 2 levels of nested for loops and the line within the innermost level take constant time\\
the line within the innermost level take constant time since values at $memo[i]$ and $P[i]$ are readily available to be fetched and max() function only involves a constant time comparison\\
\begin{align*}
    T(n) &= (\text{number of loop iterations})(\text{cost of innermost level})\\
    &= (\text{number of loop iterations})\\
    &= \sum_{i = 1}^n i\\
    &\text{/* arithmetic series with } a_1 = 1, a_n = n \text{ */}\\
    &= \frac{n(a_1 + a_n)}{2}\\
    &= \frac{n(1 + n)}{2}\\
    &= \frac{n}{2} + \frac{n^2}{2}\\
    &= \frac{n^2}{2} + \frac{n}{2}\\
    &= \Theta(n^2)\\
\end{align*}
\newpage
\noindent
Problem 3\\
(a)
\begin{align*}
canreach(n) &= canreach(n - 1) \vee canreach(n - 2) \vee ... \vee canreach(n - A[n])\\
&= \bigvee_{i = 1}^{A[n]} canreach(n - i)\\
\text{/* } \vee \text{is or */}
\end{align*}
\\
\\
\\
(b)
\[canreach(1) = true\]
/* when the frog is at $A[1]$, the frog reaches the destination
(so the frog can reach the 1st index of the array), the function returns true */
\\
\\
\newpage
\noindent
(c)
\begin{lstlisting}[language = c]
#include <stdbool.h>

bool canreach(int * A, int n)
{
    bool memo[n + 1];
    memo[1] = true;
    for(int i = 1; i <= n; i++)
    {
        if(A[i] == 0)
        {
            memo[i] = false;
            continue;
        }
        else if(i - A[i] <= 1)
        {
            memo[i] = true;
            continue;
        }
        else
        {
            for(int j = 1; j <= A[i] && i - j >= 1; j++)
            {
                if(memo[i - j])
                {
                    memo[i] = memo[i - j];
                    continue;
                }
            }
        }
    }
    return memo[n];
}
\end{lstlisting}
\leavevmode
\\
\\
\\
let $T(n)$ be the running time of canreach() in the worst case
\[T(n) = O(n^2)\]
since there are 2 levels of nested for loops and the lines within the innermost level take constant time\\
the lines within the innermost level take constant time since values at $memo[i - j]$ and $A[i]$ are readily available to be fetched\\
in the worst case, $A[i] = i - 1$\\
\begin{align*}
    T(n) &= (\text{number of loop iterations})(\text{cost of innermost level})\\
    &= (\text{number of loop iterations})\\
    &= \sum_{i = 1}^n A[i]\\
    &= \sum_{i = 1}^n (i - 1)\\
    &= \sum_{i = 1}^n i - \sum_{i = 1}^n 1\\
    &= - (n - 1 + 1)1 + \sum_{i = 1}^n i\\
    &= - n + \sum_{i = 1}^n i\\
    &\text{/* arithmetic series with } a_1 = 1, a_n = n \text{ */}\\
    &= -n + \frac{n(a_1 + a_n)}{2}\\
    &= -n + \frac{n(1 + n)}{2}\\
    &= -n + \frac{n}{2} + \frac{n^2}{2}\\
    &= \frac{n^2}{2} - \frac{n}{2}\\
    &= O(n^2)\\
\end{align*}
\\
\\
\\
Problem 4\\
(a)
\[makechange(8) = 5\]
$i = 8$: There are 5 ways:\\
1 + 1 + 1 + 1 + 1 + 1 + 1 + 1 (use 8 pennies),\\
1 + 1 + 1 + 5 (3 pennies and 1 nickel),\\
1 + 1 + 5 + 1 (2 pennies, 1 nickel, and 1 penny),\\
1 + 5 + 1 + 1 (1 penny, 1 nickel, and 2 pennies),\\
5 + 1 + 1 + 1 (1 nickel and 3 pennies)\\
\\
\\
\\
\[makechange(9) = 6\]
$i = 9$: There are 6 ways:\\
1 + 1 + 1 + 1 + 1 + 1 + 1 + 1 + 1 (use 9 pennies),\\
1 + 1 + 1 + 1 + 5 (4 pennies and 1 nickel),\\
1 + 1 + 1 + 5 + 1 (3 pennies, 1 nickel, and 1 penny),\\
1 + 1 + 5 + 1 + 1 (2 pennies, 1 nickel, and 2 pennies),\\
1 + 5 + 1 + 1 + 1 (1 penny, 1 nickel and 3 pennies)\\
5 + 1 + 1 + 1 + 1 (1 nickel and 4 pennies)\\
\\
\\
\\
(b)
\begin{align*}
&\forall n \geq 10,\; n \in \mathbb{N}\\
&makechange(n)\\
&= makechange(n - 1) + makechange(n - 5) + makechange(n - 10)\\ 
\end{align*}
\\
\\
\newpage
\noindent
(c)
\begin{lstlisting}[language = c]
int makechange(int n)
{
    int memo[n + 1];
    for(int i = 1; i <= n; i++)
    {
        if(i <= 4)
            memo[i] = 1;
        else if(i == 5)
            memo[i] = 2;
        else if(n >= 5)
        {
            memo[i] = memo[i - 1] 
                    + memo[i - 5];
        }
        else if(n >= 10)
        {
            memo[i] = memo[i - 1] 
                    + memo[i - 5] 
                    + memo[i - 10];
        }
    }
    return memo[n];
}
\end{lstlisting}
\leavevmode
\\
\\
\\
let $T(n)$ be the running time of makechange()
\[T(n) = \Theta(n)\]
since there is 1 level of for loop and the lines within the for loop take constant time\\
the lines within the for loop take constant time since values at $memo[i],\;memo[i - 1],\;memo[i - 5],\;memo[i - 10]$ are readily available to be fetched and the if statements are constant time comparisons\\
\begin{align*}
    T(n) &= (\text{number of loop iterations})(\text{cost of innermost level})\\
    &= (\text{number of loop iterations})\\
    &= \sum_{i = 1}^n 1\\
    &= n - 1 + 1\\
    &= n\\
    &= \Theta(n)\\
\end{align*}
\end{document}