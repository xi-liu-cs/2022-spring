\documentclass[12pt,border=4pt,multi]{article}%\documentclass[tikz,border=4pt,multi]{article}
\usepackage{lingmacros}
\usepackage{tree-dvips}
\usepackage{amssymb} %for mathbb{}
\usepackage[dvipsnames]{xcolor}
\usepackage{forest}
\usepackage{amsmath} %for matrices
\usepackage{xeCJK}
\usepackage{tikz}
\usepackage[arrowdel]{physics}
\usepackage{graphicx}
\usepackage{wrapfig}
\usepackage{listings}
\usepackage{pgfplots, pgfplotstable}
\graphicspath{{./img}} %specify the graphics path to be relative to the main .tex file, denoting the main .tex file directory as ./
\newcommand\Myperm[2][^n]{\prescript{#1\mkern-2.5mu}{}P_{#2}}
\newcommand\Mycomb[2][^n]{\prescript{#1\mkern-0.5mu}{}C_{#2}}
\definecolor{orchid}{rgb}{0.7, 0.4, 1.1}

\begin{document}

\section*{Xi Liu, xl3504, Midterm Exam}
\leavevmode
\\
\\
\\
\begin{verbatim}
 You may use the textbook, slides, and any notes you have.
But you may not use the internet. 
 You may NOT use communication tools 
to collaborate with other humans. This includes 
but is not limited to G-Chat, Messenger, E-mail, etc. 
 Do not try to search for answers
on the internet it will show in your answer and you will 
earn an immediate grade of 0. 
 Anyone found sharing answers 
or communicating with another student
during the exam period will earn an immediate grade of 0. 
 "I understand the ground rules and
agree to abide by them. I will not share answers 
or assist another student during 
this exam, nor will I seek assistance from another 
student or attempt to view their answers." 
\end{verbatim}
\newpage
\noindent
Problem 1\\
a.\\
yes, if the prediction is correct, then time is saved by not waiting for the conditional jump; also, running shorter jobs first may reduce turnaround time\\
\\
\\
\\
b.
\begin{figure}[h!]
	\centering
	\includegraphics[width=1.1\textwidth, height=0.6\textwidth]{1b} %img size
\end{figure}\\
\\
\\
\\
c.\\
false, different processes have different virtual address spaces. for each distinct process, the operating system maintains a different page table for the process, so different processes' virtual pages map to different physical page frames; conventionally, a page size ($2^{12} = 4096$ bytes) is a lot bigger than a cache line size ($2^6 = 64$ bytes), so 2 different physical pages cannot be contained in a same cache line, so there is no false sharing among processes that do not share memory\\
\newpage
\noindent
Problem 2\\
a.\\
yes\\
$span_1 = \text{time of }\{A \rightarrow B \rightarrow E \rightarrow F \rightarrow G\}= T_{\infty} = 40\text{ nanoseconds}$\\
$span_2 = \text{time of }\{A \rightarrow C \rightarrow E \rightarrow F \rightarrow G\}= T_{\infty} = 40\text{ nanoseconds}$\\
$span_3 = \text{time of }\{A \rightarrow D \rightarrow F \rightarrow G\} = T_{\infty} = 40 \text{ nanoseconds}$\\
\\
\\
\\
b.\\
\[\text{minimum number of cores for maximum speedup} = 3\]
\[T_1 = \sum_{i \in \{A, ..., G\}} (\text{time of task}_i) = (10 + 5 + 5 + 10 + 5 + 10 + 10)ns = 55\text{ nanoseconds}\]
\[speedup = \frac{T_{serial}}{T_{parallel}} = \frac{T_1}{T_3} = \frac{55 \text{ nanoseconds}}{40 \text{ nanoseconds}} = 1.375\]
\begin{center}
\begin{tabular}{cccc}
 time interval (ns) & process 1 & process 2 & process 3\\
 $[0, 10]$ & $A$ & &\\
 $[10, 15]$ & $B$ & $C$ & $D$\\
 $[15, 20]$ & $E$ & $D$ &\\
 $[20, 30]$ & $F$ & &\\
 $[30, 40]$ & $G$ & &\\
\end{tabular}
\end{center}
\leavevmode
\\
\\
\\
c.\\
yes, we are facing load imbalance since process 2 is idle for time interval $[0, 10] \cup [20, 30] \cup [30, 40]$; process 3 is idle for time interval $[0, 10] \cup [15, 20] \cup [20, 30] \cup [30, 40]$\\
\\
\\
\\
d.\\
let $p$ be the number of cores available to use\\
\[efficiency = \frac{speedup}{p} = \frac{1.375}{3} = 0.458\overline{3}\]
\\
\\
\\
e.\\
execution time = (instruction count) $\cdot$ (cycles per instruction) $\cdot$ (cycle time)\\
\[ET = IC \cdot CPI \cdot CT\]
\[CPI = \frac{ET}{IC \cdot CT}\]
\[ET = T_1 = \sum_{i \in \{A, ..., G\}} (\text{time of task}_i) = 55\text{ nanoseconds} = 55 \cdot 10^{-9} seconds\]
\[IC = 7 \cdot 100 \text{ instructions} = 700 \text{ instructions}\]
\[frequency = 4 GHz = 4 \cdot 10^9 Hertz = 4 \cdot 10^9 \frac{cycle}{second}\]
\[CT = \frac{1}{frequency} = \frac{1}{4 \cdot 10^9} = 2.5 \cdot 10^{-10} \frac{second}{cycle}\]
\[CPI = \frac{55 \cdot 10^{-9} seconds}{(700 \text{ instructions})(2.5 \cdot 10^{-10} \frac{second}{cycle})} = 0.314286 \frac{cycle}{instruction}\]
\newpage
\noindent
Problem 3\\
a.\\
{\large
\begin{center}
\begin{tabular}{|c|c|c|c|}
\hline
& process 0 & process 1 & process 2\\
\hline
x & 4 & 6 & 8\\
\hline
y & 8 & 8 & 2\\
\hline
z & 8 & 4 & 8\\
\hline
\end{tabular}
\end{center}
}
\leavevmode
\\
\\
\\
b.\\
without the break statement, when my\_rank = 1, the code within the original case 2 block will be executed\\
before process 1 finishes the original code block of case 1, process 0, 1, and 2 would initially behave like the version of code where ``break;" is not removed,\\
but then for process 1: after the execution of the original case 1 code block, then $x, y, z$ will be reassigned with new values such that $x = 3;\; y = 2,\; z = 1$; then the last 2 additional MPI\_Bcast() functions calls can cause the program to hang since all of the other processes in the communicator have finished their MPI\_Bcast()\\
\\
\\
\\
c.\\
yes\\
with hyperthreading, 1 physical core can be used as 2 virtual or logical cores by the operating system, allowing process 0 and process 1 to both execute on the same core and at the same time\\
\\
\\
\\
\begin{center}
\begin{tabular}{ccc}
 time interval (ns) & process 1 & process 2\\
 $[0, 10]$ & $A$ & \\
 $[10, 15]$ & $B$ & $D$\\
 $[15, 20]$ & $C$ & $D$\\
 $[20, 25]$ & & $E$\\
 $[25, 35]$ & & $F$\\
 $[35, 45]$ & & $G$\\
\end{tabular}
\end{center}
\end{document}