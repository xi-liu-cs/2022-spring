\documentclass[12pt,border=4pt,multi]{article}%\documentclass[tikz,border=4pt,multi]{article}
\usepackage{lingmacros}
\usepackage{tree-dvips}
\usepackage{amssymb} %for mathbb{}
\usepackage{xcolor}
\usepackage{forest}
\usepackage{amsmath} %for matrices
\usepackage{xeCJK}
\usepackage{tikz}
\usepackage[arrowdel]{physics}
\usepackage{graphicx}
\usepackage{wrapfig}
\graphicspath{{./images}} %specify the graphics path to be relative to the main .tex file, denoting the main .tex file directory as ./
\usepackage{esint}
\newcommand\Myperm[2][^n]{\prescript{#1\mkern-2.5mu}{}P_{#2}}
\newcommand\Mycomb[2][^n]{\prescript{#1\mkern-0.5mu}{}C_{#2}}
\definecolor{mypink1}{rgb}{0.75, 0.15, 0.5}

\begin{document}

\section*{Xi Liu, xl3504, Homework 2}

\begin{verbatim}

1. 1.

"prev" = "previous"
"ret" = "return"
"addr" = "address"
 
	|    ...	|
	|  prev %rbp before main	| 
	|  line after L7, or ret addr (current %rip, addr of current instruction line) |
	|  prev %rbp before comp | <- frame pointer (%rbp)
	|  0, or value of e	|
	|  0, or value of f	|  <- stack pointer (%rsp)



1.2.

 
	|    ...	|
	|  prev %rbp before main	| 
	|  line after L7, or ret addr (current %rip, addr of current instruction line) |
	|  prev %rbp before comp | <- frame pointer (%rbp)
	|  0, or value of e	|
	|  0, or value of f	|  <- stack pointer (%rsp)

%rip would correspond to L5

1.3.

	|    ...	|
	|  prev %rbp before main	| 
	|  line after L7, or ret addr (current %rip, addr of current instruction line) |
	|  prev %rbp before comp | 
	|  0, or value of e	|
	|  0, or value of f	|  
	|  L5, or ret address (current %rip, address of current line of instruction) | 
	|  prev %rbp before mul | <- frame pointer (%rbp) <- stack pointer (%rsp)



1.4
	
	|    ...	|
	|  prev %rbp before main	| 
	|  line after L7, or ret addr (current %rip, addr of current instruction line) |
	|  prev %rbp before comp | 
	|  0, or value of e	|
	|  0, or value of f	|  
	|  L5, or ret addr (current %rip, addr of current instruction line) | <- frame pointer (%rbp) <- stack pointer (%rsp)

%rip would correspond to L1



2.1

    node_t *
    find_insert_pos(node_t *head, node_t *node)
    {
        if (head == NULL) return NULL;

        node_t *ret = NULL;

        // 2.1 your code here
        if(node -> id < head -> id)
        {
            return NULL;
        }

        ret = head;
        while(ret)
        {
            if( ret -> id >= node -> id ) 
            {   
                return ret;
            }
            if(ret -> next -> id > node -> id) 
            {
                return ret;
            }
            ret = ret -> next;
        }

        return ret;
    }







2.2

   node_t * 
    list_insert(node_t *head, node_t *node)
    {
        if (head == NULL) return node;

        // find the proper position to insert this node pair.
        node_t *pos = find_insert_pos(head, node);

        // 2.2 your code here

        if(!pos)
        {
            node -> next = head;
            head = node;
        }
        else
        {
            node -> next = pos -> next;
            pos -> next = node;
        }
    }




3.1

	i.  echo hello
	ii. echo hello $world
	iii. echo hello
	iv. hello
	v. -bash: syntax error near unexpected token `echo'



3.2 
	i. hello world
	ii. no printed message seen
	iii. hello world



3.3
	i. 
		a
		b

	ii.
		a
		b

	iii 
		[1]  1941
		b
		a		
	
3.4
on my computer, below seems to output correctly:
first part
	cat members.txt | head -n100| cut -d ':' -f2
second part
	cat members.txt | sort -f | head -n100| cut -d ':' -f2 | tee names.txt
	

grep might be needed:
first part
	cat members.txt |  grep "^Name:[a-zA-Z']\+$" | head -n100| cut -d ':' -f2
second part
	cat members.txt |  grep "^Name:[a-zA-Z']\+$" | sort -f | head -n100| cut -d ':' -f2 | tee names.txt

\end{verbatim}

\end{document}
	
	